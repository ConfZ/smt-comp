\documentclass[12pt]{article}

\usepackage{color}
\usepackage{times}
\usepackage{fullpage}
\usepackage{hyperref}
\usepackage{xspace}
\usepackage{enumitem}

%\usepackage{draftwatermark}

\hyphenation{data-types}

\newcommand{\akey}[1]{\textbf{#1}}
\newcommand{\bkey}[1]{\textbf{\texttt{#1}}}

\newcommand{\rem}[1]{\textcolor{red}{[#1]}}
\newcommand{\todo}[1]{\rem{TODO #1}}
\newcommand{\an}[1]{\rem{#1 -- aina}}
\newcommand{\ah}[1]{\rem{#1 -- antti}}
\newcommand{\lh}[1]{\rem{#1 -- liana}}
\newcommand{\gr}[1]{\rem{#1 -- giles}}

\newcommand{\maintrack}{Single Query Track\xspace}
\newcommand{\inctrack}{Incremental Track\xspace}
\newcommand{\ucoretrack}{Unsat Core Track\xspace}
\newcommand{\mvaltrack}{Model Validation Track\xspace}
\newcommand{\challtrack}{Industry Challenge Track\xspace}

\newcommand{\rationale}[1]{\hskip .5em{\textit{Rationale:} #1}\xspace}

\let\OLDthebibliography\thebibliography
\renewcommand\thebibliography[1]{
  \OLDthebibliography{#1}
  \setlength{\parskip}{2pt}
  \setlength{\itemsep}{1.5pt plus 0.3ex}
}

\urlstyle{same}

%%%%%%%%%%%%%%%%%%%%%%%%%%%%%%%%%%%%%%%%%%%%%%%%%%%%%%%%%%%%%%%%%%%%%%%%%%%%%%%%

\begin{document}

\date{\small This version revised \the\year-\the\month-\the\day}

\title{14th International Satisfiability Modulo Theories Competition
  (SMT-COMP 2019): Rules and Procedures}

% [morgan] do our own layout of authors; the four-author layout spacing
% was screwed up...
\def\doauthor#1{{%
  \hsize.5\hsize \advance\hsize by-1cm %
  \def\\{\hss\egroup\hbox to\hsize\bgroup\strut\hss}%
  \vbox{\hbox to\hsize\bgroup\strut\hss#1\hss\egroup}}}%

\def\header#1{\medskip\noindent\textbf{#1}}

\author{%
Liana Hadarean \\
Amazon \\
USA \\
{\small\href{mailto:hadarean@amazon.com}{\texttt{hadarean@amazon.com}}}\\
\and
Antti Hyvarinen \\
Universita della Svizzera italiana \\
Switzerland \\
{\small\href{mailto:antti.hyvaerinen@usi.ch}{\texttt{antti.hyvaerinen@usi.ch}}} \\
\and
Aina Niemetz \\
Stanford University\\
USA\\
{\small\href{mailto:niemetz@cs.stanford.edu}{\texttt{niemetz@cs.stanford.edu}}}\\
\and
Giles Reger \\
University of Manchester \\
UK \\
{\small\href{mailto:giles.reger@manchester.ac.uk}{\texttt{giles.reger@manchester.ac.uk}}} \\
\\
}

\maketitle

\noindent Comments on this document should be emailed to the SMT-COMP
mailing list (see below) or, if necessary, directly to the organizers.

%%%%%%%%%%%%%%%%%%%%%%%%%%%%%%%%%%%%%%%%%%%%%%%%%%%%%%%%%%%%%%%%%%%%%%%%%%%%%%%%

\section{Communication}

Interested parties should subscribe to the SMT-COMP mailing list.
Important late-breaking news and any necessary clarifications and
edits to these rules will be announced there, and it is the primary
way that such announcements will be communicated.

\begin{itemize}
\item SMT-COMP mailing list:
  \href{mailto:smt-comp@cs.nyu.edu}{\textrm{smt-comp@cs.nyu.edu}}
\item Sign-up site for the mailing list:
  \url{http://cs.nyu.edu/mailman/listinfo/smt-comp}
\end{itemize}

\noindent Additional material will be made available at the
competition web site, \url{http://www.smtcomp.org}.

%%%%%%%%%%%%%%%%%%%%%%%%%%%%%%%%%%%%%%%%%%%%%%%%%%%%%%%%%%%%%%%%%%%%%%%%%%%%%%%%

\newpage

\section{Important Dates}
\label{sec:important}

\begin{description}
\item[March~1] Deadline for new benchmark contributions.
\item[May~1] Final versions of competition tools (e.g., benchmark
  scrambler) are made available.  Benchmark libraries are frozen.
\item[May~19] Deadline for first versions of solvers (for all tracks),
  including information about which tracks and divisions are being
  entered, and magic numbers for benchmark scrambling.
\item[June~2] Deadline for final versions of solvers, including
  system descriptions.
\item[\todo{}] Opening value of NYSE Composite Index used to complete
  random seed for benchmark scrambling.
\item[July~7/8] SMT Workshop; end of competition, presentation of
  results.
\end{description}

%%%%%%%%%%%%%%%%%%%%%%%%%%%%%%%%%%%%%%%%%%%%%%%%%%%%%%%%%%%%%%%%%%%%%%%%%%%%%%%%

\section{Introduction}

\todo{does this paragraph need rewriting?}
The annual Satisfiability Modulo Theories Competition~(SMT-COMP) is
held to spur advances in SMT solver implementations on benchmark
formulas of practical interest.  Public competitions are a well-known
means of stimulating advancement in software tools.  For example, in
automated reasoning, the CASC and SAT competitions for first-order and
propositional reasoning tools, respectively, have spurred significant
innovation in their fields~\cite{leberre+03,PSS02}.  More information
on the history and motivation for SMT-COMP can be found at the
competition web site, \url{http://www.smtcomp.org}, and in reports on
previous
competitions~(\cite{SMTCOMP-JAR,SMTCOMP-FMSD,BDOS08,SMTCOMP-2008,CDW14,SMTCOMP-2012,CSW15}).

SMT-COMP~2019 is part of the SMT Workshop~2019
(\url{http://smt2019.galois.com/}),
which is affiliated with SAT~2019 (\url{http://sat2019.tecnico.ulisboa.pt/}).
The SMT Workshop will include a block of time to present the results of the
competition.

Accordingly, researchers are highly encouraged to submit both new
benchmarks and new or improved solvers to raise the level of
competition and advance the state-of-the-art in automated SMT problem
solving.

SMT-COMP 2019 will have five tracks: the \maintrack (previously: Main Track),
the \inctrack (previously: Application Track), the \ucoretrack,
the (new) \challtrack, and the (new)
\mvaltrack.

Within each track there are multiple divisions, where each division
uses benchmarks from a specific SMT-LIB logic (or group of logics).
We will recognize winners as measured by number of benchmarks solved
(taking into account the weighting detailed in
Section~\ref{sec:scoring}); we will also recognize solvers based on
additional criteria.

The rest of this document, revised from the previous
version,\footnote{Earlier versions of this document include
  contributions from Clark Barrett, Roberto Bruttomesso, David Cok,
  Sylvain Conchon, David D{\'e}harbe, Morgan Deters, Alberto Griggio,
  Matthias Heizmann, Aina Niemetz, Albert Oliveras, Giles Reger, Aaron Stump,
	and Tjark Weber.}  describes the rules and competition procedures for
SMT-COMP~2019.
The principal changes from the previous competition rules are the following: \todo{Aina in charge of this section}
\begin{itemize}
  \item {\bf Mandatory System Descriptions.}
    SMT-COMP entrants are now required to provide a short (1–2 pages)
    description of the system.
    \rationale{The main incentive for this
    change is twofold.  First, we want to improve transparency when submitted
    solvers are wrapper tools according to the rules of the competition.
    Second, we want to encourage documenting technical improvements that
    lead to the current results.}

  \item {\bf Naming Convention for Derived Tools.} \todo{}

  \item {\bf Renaming of Tracks.} This year, the track previously known as
    `Main Track` will be renamed to \emph{\maintrack}, and the previous
    `Application Track' will be renamed to \emph{\inctrack}.
    \rationale{We believe the current names are misleading, as the
    previous `Main Track' also contains problems coming from applications.
    Additionally, having it called `Main' de-emphasizes the importance of the
    other tracks and use cases of SMT.}

  \item {\bf Incremental Track.}
    In previous years, benchmarks were not eligible for the \inctrack
    if their first \akey{check-sat} command had unknown status.  Similarly, the
    trace executor used to send commands from the benchmark to the solver's
    standard input channel stopped execution at the first \akey{check-sat}
    command with unknown status. This year, benchmarks whose first
    \akey{check-sat} command has unknown status are eligible for the
    \inctrack and trace execution is only stopped before the solver
    completes a benchmark if the execution runs into the time limit.
    \rationale{The previous limitations were imposed by the trace
    executor, which now has been extended to support executing solvers on
    benchmarks beyond \akey{check-sat} commands with unknown status.}

  \item {\bf New \mvaltrack for QF\_BV.}
    This year, we introduce a new experimental \mvaltrack for the QF\_BV logic.
    The benchmarks for this track will include all eligible \emph{non-incremental}
    QF\_BV benchmarks with known satisfiable status in SMT-LIB.
    Participating solvers are required to support the \akey{get-model} command.
    \rationale{In many SMT applications, model generation is an essential
    feature.  Previously, none of the SMT-COMP tracks required model
    generation. One of the challenges is that the model format is not
    consistent across different solvers. While imposing a standard over all
    logics is challenging, there are several logics (e.g., QF\_BV) where it is
    straightforward.  In the future we hope to expand this track to other
    logics as a way of pushing for model standardization.  Since the QF\_BV
    division was the one with the largest number of participants in previous
    years,  we hope for a high number of participants in this track.  }

  \item {\bf New \challtrack.}
    This year, we introduce a new \challtrack.  This track will contain
    new challenging SMT-LIB benchmarks (with an emphasis on industrial
    applications) that are either unsolved or unsolved within some reasonable
    time limit, as indicated by the benchmark submitters.  We will also include
    benchmarks nominated by the community as challenging and of interest. In
    this track, solvers will run with a significantly longer time limit.
    \rationale{ Different application domains require different time limits.
    For example software verification traditionally requires much lower time
    limits, compared to hardware verification.  To reward solvers optimized for
    different use cases, we propose this new track and a new score for
    benchmarks solved within a very low time limit (see item ``New Scores in
    the \maintrack'').  } See ~\ref{sec:exec:industry-challenge} for the logics
    included in \challtrack for this year.

  \item {\bf Time Limit.} The time limit per solver/benchmark pair is
    anticipated to be at most 40 minutes in the \maintrack, \inctrack,
    \ucoretrack and \mvaltrack.  For the \challtrack, it is anticipated to be
    at most 720 (12 hours) minutes.
    \rationale{In 2017 and 2018, the time limit for the Main Track was
    reduced to 20 minutes (down from 40 minutes in earlier years) to cope with
    the inclusion of a large number of benchmarks with unknown status.  This
    year, the reduced number of benchmarks selected per division allows us to
    increase the time limit back to 40 minutes for the \maintrack.  For the
    \challtrack, 12 hours seem to be a reasonable compromise for the
    expected number of participants and benchmarks submitted to this track.}

  \item {\bf Benchmark Selection.} Since 2015, the competition evaluated all
    solvers on all eligible benchmarks in SMT-LIB.  This year, we will
    use an alternative benchmark selection scheme that selects a subset
    of the eligible benchmarks. The benchmark selection will be random,
    but will guarantee inclusion of newly submitted benchmarks.  The upper
    bound for the size of each division is not fixed, but
    depends on the size of the corresponding logic.  Prior to this
    selection, we will remove all benchmarks in a division that were
    solved by all solvers (including non-competing solvers) in this
    division in under 1 second during last year's competition.
    \rationale{Evaluating solvers on all eligible benchmarks in SMT-LIB
    makes results more predictable and seems to be more of an evaluation
    than a competition.  In the Main track of last year's competition,
    78\% of the 258,741 benchmarks were solved by all supported solvers
    within the time limit (71\% within 1 second). In 7 (out of 46)
    logics, over 99\% of benchmarks were solved by all solvers.
    Removing `easy' (or at least `unsurprising') benchmarks attempts to
    shift the focus towards challenging benchmarks, in particular since
    we now use an alternative benchmark selection scheme.  It can be
    argued that the size of a logic in SMT-LIB can be seen as an
    indicator of its relevance.  We thus do not use a fixed upper bound
    for the size of a division in order to reflect the suggested
    importance of a division.}

  \item {\bf Scoring Scheme.}
    This year, we will abandon the weighted scoring scheme that was introduced
    for the Main Track and \ucoretrack in 2016. We will fall back to the
    scoring scheme based on the number of solved instances that was last used
    in 2015.
    \rationale{ Since 2016, the competition has used a scoring scheme based on
    benchmark weights in order to de-emphasize large benchmark families.  The
    scoring scheme is fairly complicated and not necessarily intuitive.  It
    further makes comparing results in papers with results from the competition
    more difficult.  A recent analysis of competition data from 2015-2018 for a
    report on the competition of these years (currently under submission to
    JSAT) suggests that benchmark families do not have a significant impact on
    the (weighted) scores  When determining winners for each division using
    the scoring scheme of 2015, the winners for only 7 divisions (out of 139)
    over all three years would have changed.  While producing these results we
    further noticed that the description of what constitutes a benchmark family
    (as stated in the rules documents) had been incorrectly interpreted by the
    scoring scripts in 2016-2018. After fixing this misinterpretation, only one
    single division winner changes - in 2017 AUFNIRA should have been won by
    CVC4 and not Vampire.}

  \item {\bf New Scores in the \maintrack.}
    This year, additionally to the separate scores given for sequential and
    parallel performance, we will reward three new scores in the \maintrack.
    The \emph{24-second score} will reward solving performance within a time
    limit of 24 seconds (wall clock time), the \emph{sat score} will reward
    performance on satisfiable instances, and the \emph{unsat score} will
    reward performance on unsatisfiable instances.
    \rationale{ Different application domains of SMT typically impose a wide
    range of time limits (from hours to seconds).  Previous time limits used in
    the competition were the same for all benchmarks and thus agnostic to the
    application domain of a benchmark.  The new \challtrack and the new
    24-second score try to address use cases on both extreme ends of the
    spectrum.  Further, in many cases an SMT application produces either mainly
    satisfiable or mainly unsatisfiable queries to the SMT solver.  The new sat
    and unsat scores are intended to reward solvers that implement specialized
    techniques and optimizations for either case.}

  \item {\bf Do Not Run Non-Competitive Divisions.}
    This year, we will not run non-competitive divions.
    \rationale{Evaluating solvers in non-competitive divisions is more in the
    spirit of an evaluation than a competition.}

  \item {\bf Experimental Strings Division.}
    \todo{point to the current draft of the standard as the set of rules
    solvers have to agree on}
    The competition will feature an experimental divisions for benchmarks that
    use strings.
    \rationale{Corresponding theories and benchmarks are expected to be added
    to SMT-LIB in the near future.}

  \item {\bf Competition-Wide Recognitions.} \todo{}

  \item {\bf Problems with Evaluation Scripts.} \todo{Explain what was the
    problem and what has been fixed, talk about notion of benchmark family.}

\end{itemize}

%%%%%%%%%%%%%%%%%%%%%%%%%%%%%%%%%%%%%%%%%%%%%%%%%%%%%%%%%%%%%%%%%%%%%%%%%%%%%%%%

\section{Entrants}
\label{sec:entrants}

\header{SMT Solver.}
%
A SMT solver that can enter SMT-COMP is a tool that can
determine the (un)satisfiability of benchmarks from the SMT-LIB benchmark library
(\url{http://www.smt-lib.org/benchmarks.shtml}).

\header{Wrapper Tool.}
%
A \emph{wrapper tool} is defined as any solver that calls one or more other SMT
solvers (the \emph{wrapped solvers}). Its system description \textbf{must} explicitly acknowledge any
solvers that it wraps.  It \emph{should} further make clear technical
innovations by which the wrapper tool expects to improve on the wrapped
solvers.

\header{Derived Tool.}
%
A \emph{derived tool} is defined as any solver that is \emph{based on and
extends} another SMT solver (the \emph{base solver}).  Its system description
\textbf{must} explicitly acknowledge
the solver it is based on and extends.  It \emph{should} further make clear
technical innovations by which the derived tool expects to improve on the
original solver.  A derived tool should follow the \emph{naming convention}
{[name of base solver]-[my solver name]}.

\header{SMT Solver Submission.}
%
An entrant to SMT-COMP is a solver submitted by its authors using
the StarExec (\url{http://www.starexec.org}) service.

\header{Solver execution.}
%
The StarExec execution
service enables members of the SMT research community to run solvers
on jobs consisting of benchmarks from the SMT-LIB benchmark library.
Jobs are run on a shared computer cluster.  The execution service is
provided free of charge, but requires registration to create a
login account.  Registered users may then upload solvers to
run, or may run public solvers already uploaded to the service.
Information about how to configure and upload a solver is contained in
the StarExec user guide,
\url{https://wiki.uiowa.edu/display/stardev/User+Guide}.

\header{Participation in the Competition.}
%
For participation in SMT-COMP, a solver must be uploaded to StarExec
and made publicly available.  StarExec supports solver configurations;
for clarity, \emph{each submitted solver must have one configuration
  only}.  Moreover, the organizers must be informed of the solver's
presence \emph{and the tracks and divisions in which it is
  participating} via the web form at
\begin{center}
  \rem{submission form url}
\end{center}
A submission \textbf{must} also include a \emph{system description} (see below)
and a \emph{32-bit unsigned integer}.
 These integer numbers, collected from all submissions, are used to seed
 competition tools.

\header{System description.}
%
As part of the submission, SMT-COMP entrants are \textbf{required} to provide a
short (1-2 pages) description of the system, which \textbf{must} explicitly
acknowledge any solver it wraps or is based on in case of a \emph{wrapper} or
\emph{derived} tool (see above).
It \emph{should} further include
\begin{itemize}[itemsep=0ex]
  \item a list of all authors of the system and their present institutional
    affiliations,
  \item the basic SMT solving approach employed,
  \item details of any non-standard algorithmic techniques as well as
    references to relevant literature (by the authors or others),
  \item in case of a \emph{wrapped} or \emph{derived tool}: details of
    technical innovations by which a wrapper or derived tool expects to improve
    on the wrapped,
    solvers or base solver
  \item appropriate acknowledgement of tools other than SMT solvers called by
    the system (e.g., SAT solvers) that are not written by the authors of the
    submitted solver, and
  \item a link to a website for the submitted tool.
\end{itemize}
System descriptions \textbf{must} be submitted \textbf{until the final solver
deadline}, and will be made publicly available on the competition website.

\header{Multiple versions.}
%
The intent of the organizers is to promote as wide a comparison among
solvers and solver options as possible.  However, if the number of
solver submissions is too large for the computational resources
available to the competition, the organizers reserve the right not to
accept multiple versions of solvers from the same solver team.

\header{Other solvers.}
%
The organizers reserve the right to include other solvers of interest
(such as entrants in previous SMT competitions) in the competition,
e.g., for comparison purposes.

\header{Attendance.}
%
Submitters of an SMT-COMP entrant are not required (but encouraged) to be
physically present at the competition or the SMT Workshop to participate or
win.

%%%%%%%%%%%%%%%%%%%%%%%%%%%%%%%%%%%%%%%%%%%%%%%%%%%%%%%%%%%%%%%%%%%%%%%%%%%%%%%%

\subsection*{Deadlines}

SMT-COMP entrants must be submitted via StarExec (solvers) \emph{and}
the above web form (accompanying information) until the end of
{\bf May~19, 2019} anywhere on earth.
After this date \emph{no new entrants} will be accepted.
However, updates to existing entrants on StarExec
will be accepted until the end of {\bf June~2, 2019} anywhere on earth.

We strongly encourage participants to use this grace period
\emph{only} for the purpose of fixing any bugs that may be discovered,
and not for adding new features, as there may be no opportunity to do
extensive testing using StarExec after the initial deadline.

The solver versions that are present on StarExec at the conclusion of
the grace period will be the ones used for the competition.  Versions
submitted after this time will not be used.  The organizers reserve
the right to start the competition itself at any time after the open
of the New York Stock Exchange on the day after the final solver
deadline.

These deadlines and procedures apply equally to all tracks of the
competition.



%%%%%%%%%%%%%%%%%%%%%%%%%%%%%%%%%%%%%%%%%%%%%%%%%%%%%%%%%%%%%%%%%%%%%%%%%%%%%%%%

\section{Execution of Solvers}

Solvers will be publicly evaluated in all tracks and divisions into
which they have been entered.  All results of the competition will be
made public.
\todo{Giles: Include licence information - minimum requirement: licence must allow to run the binary for evaluation purposes}
\an{intro of the section maybe, we should add there that solvers will be archived on the website and so on. Interestingly, we replied to the reviewer in the report that the solvers are publicly available in the StarExec space when asked about reproducability, but in the past it was never defined under which terms.}


\subsection{Logistics}
\label{sec:logistics}

\header{Dates of Competition.}
%
The bulk of the computation will take place during the weeks leading
up to SMT 2019.  Intermediate results will be regularly posted to the
SMT-COMP website as the competition runs.
%
The organizers reserve the right to prioritize certain competition
tracks or divisions to ensure their timely completion, and in
exceptional circumstances to complete divisions after the SMT
Workshop.


\header{Competition Website.}
The competition website (\url{www.smtcomp.org}) will be used as the main form
of communication for the competition. The website will be used to post updates,
link to these rules and other relevant information (e.g. the benchmarks), and
to announce the results. We also use the website to archive previous
competitions. Starting from 2019 we will include the submitted solvers in this
archive to allow reproduction of the competition results in the future.
%

\header{Tools.}
The competition uses a number of tools/scripts to run the competition. In the
following, we briefly describe these tools. Unless stated otherwise, these
tools are found at \url{https://github.com/SMT-COMP/smt-comp/tree/master/tools}.
{\color{red} Some tools have not been released yet.}
\begin{itemize}
  \item \textbf{Benchmark Selection.} We use a script to implement the
    benchmark selection policy described on page~\pageref{benchmark-selection}.
    It takes a seed for the random benchmark selection. The same seed is used
    for all tools requiring randomisation.
  \item \textbf{Scrambler.} This tool is used to scramble benchmarks during the
    competition to ensure that tools do not rely on syntactic features to
    identify benchmarks. The scrambler can be found at
    \url{https://github.com/SMT-COMP/scrambler}.
  \item \textbf{Trace Executor.} This tool is used in the \inctrack to emulate
    an on-line interaction between an SMT solver and a client application and
    is available at \url{https://github.com/SMT-COMP/trace-executor}
  \item \textbf{Postprocessors.} These are used by StarExec to translate the
    output of tools to the format required for scoring. All postprocessors (per
    track) are available at \url{https://github.com/SMT-COMP/postprocessors}.
  \item \textbf{Scoring.} We use a script to implement the scoring computation
    described on in Section~\ref{sec:scoring}. It also includes the scoring
    computations used in previous competitions (since 2015).
\end{itemize}

\header{Input and Output.}
%
In the \emph{\inctrack}, the \emph{trace executor} will send commands from an
(incremental) benchmark file to the standard input channel of the solver.  In
\emph{all other tracks}, a participating solver must read a \emph{single}
benchmark file, whose filename is presented as the first command-line argument
of the solver.

Benchmark files are in the concrete syntax of the SMT-LIB format
version~2.6, though with a \emph{restricted} set of commands.  A benchmark
file is a text file containing a sequence of SMT-LIB commands that
satisfies the following \emph{requirements}:
%
\begin{enumerate}
  \item
    \bkey{(set-option :print-success ...)}
    \begin{enumerate}
      \vspace{-1ex}
      \item In the \textbf{\maintrack} and \textbf{\ucoretrack}, there may be a
        single \akey{set-option} command.  Note that \texttt{success} outputs
        are ignored by the post-processor used by the
        competition.\footnote{SMT-LIB~2.6 requires solvers to produce a
        \texttt{success} answer after each \akey{set-logic},
        \akey{declare-sort}, \akey{declare-fun} and \akey{assert} command
        (among others), unless the option \akey{:print-success} is set to
        false.  Ignoring the \texttt{success} outputs allows for submitting
        fully SMT-LIB~2.6 compliant solvers without the need for a wrapper
        script, while still allowing entrants of previous competitions to run
        without changes.}
      \item In the \textbf{\inctrack}, the \akey{:print-success} option
      must not be disabled.  The trace executor will send an initial
        \akey{(set-option :print-success true)} command to the solver.
      \item In the \textbf{\ucoretrack}, there is a single \akey{(set-option
        :produce-unsat-cores true)} command.
    \end{enumerate}
  \item \bkey{(set-logic ...)}\\
    A (single) \akey{set-logic} command is the \emph{first} command after
    any \akey{set-option} commands.
  \item \bkey{(set-info ...)}\\
    A benchmark file may contain any number of \akey{set-info} commands.
  \item \bkey{(declare-sort ...)}\\
    A benchmark file may contain any number of \akey{declare-sort} and
    \akey{define-sort} commands.  All sorts declared or defined with these
    commands must have zero arity.
  \item \bkey{(declare-fun ...)} and \bkey{(define-fun ...)}\\
    A benchmark file may contain any number of \akey{declare-fun} and
    \akey{define-fun} commands.
  \item \bkey{(declare-datatype ...)} and \bkey{(declare-datatypes ...)}\\
    If the logic features algebraic datatypes, the benchmark file may
    contain any number of \akey{declare-datatype(s)} commands.
  \item \bkey{(assert ...)}\\
    A benchmark file may contain any number of \akey{assert} commands.  All
    formulas in the file belong in the declared logic, with any free symbols
    declared in the file.
  \item
    \bkey{:named}
    \begin{enumerate}
      \vspace{-1ex}
      \item In \textbf{all} tracks \textbf{except} the \ucoretrack,  named
        terms (i.e., terms with the \akey{:named} attribute) are \emph{not}
        used.
      \item In the \textbf{\ucoretrack}, top-level assertions may be named.
    \end{enumerate}
    \item
      \bkey{(check-sat)}
      \begin{enumerate}
        \vspace{-1ex}
        \item In \textbf{all} tracks \textbf{except} the \inctrack, there is
          \emph{exactly one} \akey{check-sat} command.
        \item In the \textbf{\inctrack}, there are one or more
        \akey{check-sat} commands.  There may also be zero or more
        \akey{push 1} commands, and zero or more \akey{pop 1} commands,
        consistent with the use of those commands in the SMT-LIB standard.
    \end{enumerate}
    \item \bkey{(get-unsat-core)}\\
    In the \textbf{\ucoretrack}, the \akey{check-sat} command (which is
      always issued in an unsatisfiable context) is followed by a single
      \akey{get-unsat-core} command.
    \item \bkey{(get-model)}\\
      In the \textbf{\mvaltrack}, the \akey{check-sat} command (which is
      always issued in a satisfiable context) is followed by a single
    \akey{get-model} command.
  \item \bkey{(exit)}\\
    It may \emph{optionally} contain an \akey{exit} command as its
    last command.  In the \textbf{\inctrack}, this command must not be
    omitted.
  \item No other commands besides the ones just mentioned may be used.
\end{enumerate}
%
The SMT-LIB format specification is available from the ``Standard''
section of the SMT-LIB website~\cite{SMT-LIB}.  Solvers will be given
formulas only from the divisions into which they have been entered.

\header{Time and memory limits.}
%
Each SMT-COMP solver will be executed on a dedicated processor of a
competition machine, for each given benchmark, up to a fixed
wall-clock time limit~$T$. The individual track description specifies
the time limit for each track. Each processor has 4 cores.  Detailed
machine specifications are available on the competition web site.

The StarExec service also limits the memory consumption of the solver
processes.  We expect the memory limit per solver/benchmark pair to be
on the order of 60\,GB.  The values of both the time limit and the
memory limit are available to a solver process through environment
variables.  See the StarExec user guide for more information.

% The time limit~$T$ will be 40~minutes of wall-clock time per solver/benchmark
% pair in the \maintrack, \inctrack and \ucoretrack, \todo{12 hours} in the
% \challtrack, and \todo{40 + 5 as in Unsat core}~minutes in the \mvaltrack \footnote{The time limit may be
% adjusted once we know the number of competition entrants and eligible
% benchmarks.}  Solvers that take more than this time limit will be killed.
% Solvers are allowed to spawn other processes; these will be killed at
% approximately the same time as the first started process.


\subsection{\maintrack (Previously: Main Track)}
\label{sec:exec:single}

\header{Input/Output.}
This track will consist of selected non-incremental benchmarks in each of
the logic divisions.  Each benchmark will be presented to the solver as its
first command-line argument.  The solver is then expected to report on its
standard output channel whether the formula is satisfiable (\texttt{sat}) or
unsatisfiable (\texttt{unsat}).  A solver may also report \texttt{unknown} to
indicate that it cannot determine satisfiability of the formula.

\header{Time Limit.}
This track will use a wall-clock time limit of 40 minutes per solver/benchmark
pair.

\header{Postprocessor.}
This track uses a StarExec post-processor (named \rem{tba}) to accumulate the
results.

\header{Aborts and unparsable output.}
%
Any \texttt{success} outputs will be ignored.  Solvers that exit
before the time limit without reporting a result (e.g., due to
exhausting memory or crashing) \emph{and} do not produce output that
includes \texttt{sat}, \texttt{unsat} or \texttt{unknown} will be
considered to have aborted.

\header{Persistent state.}
%
Solvers may create and write to files and directories during the
course of an execution, but they must not read such files back during
later executions.  Each solver is executed with a temporary directory
as its current working directory.  Any generated files should be
produced there (and not, say, in the system's \texttt{/tmp}
directory).  The StarExec system sets a limit on the amount of disk
storage permitted---typically 20\,GB.  See the StarExec user guide for
more information.  The temporary directory is deleted after the job is
complete.  Solvers must not attempt to communicate with other
machines, e.g., over the network.

\subsection{Incremental Track (Previously: Application Track)}
\label{sec:exec:app}

The incremental track evaluates SMT solvers when interacting with an
external verification framework, e.g., a model checker. This
interaction, ideally, happens by means of an online communication
between the framework and the solver: the framework repeatedly sends
queries to the SMT solver, which in turn answers either \texttt{sat}
or \texttt{unsat}.  In this interaction an SMT solver is required to
accept queries incrementally via its \emph{standard input channel}.

In order to facilitate the evaluation of solvers in this track, we will set up
a ``simulation'' of the aforementioned interaction.  Each benchmark represents
a realistic communication trace, containing multiple \akey{check-sat} commands
(possibly with corresponding \akey{push 1} and \akey{pop 1} commands). It is
parsed by a (publicly available) \emph{trace executor},
which serves the following purposes: \begin{itemize} \item it simulates the
online interaction by sending single queries to the SMT solver (through stdin);
\item it prevents ``look-ahead'' behaviors of SMT solvers;
\item it records time and answers for each command;
\item it guarantees a fair execution for all solvers by abstracting
  from any possible crash, misbehavior, etc.\ that might happen in the
  verification framework.
\end{itemize}

The disk space and memory limits for the incremental track are the
same as for the \maintrack (see Section~\ref{sec:exec:single}).

\header{Input and output.}
%
Participating solvers will be connected to a trace executor, which
will incrementally send commands to the standard input channel of the
solver and read responses from the standard output channel of the
solver.  The commands will be taken from an SMT-LIB benchmark script
that satisfies the requirements for incremental track scripts given in
Section~\ref{sec:logistics}.

Solvers must respond to each command sent by the trace executor with
the answers defined in the SMT-LIB format specification, that is, with
an answer of \texttt{sat}, \texttt{unsat}, or \texttt{unknown} for
\akey{check-sat} commands, and with a \texttt{success} answer for
other commands.

\subsection{\ucoretrack}
\label{sec:exec:unsat-core}

The \ucoretrack will evaluate the capability of solvers to
generate unsatisfiable cores (for problems that are known to be
unsatisfiable).  Solvers will be measured by the smallness of the
unsatisfiable core they return.

The SMT-LIB language accommodates this functionality by providing two
features: the ability to name top-level (asserted) formulas, and the
ability to request an unsatisfiable core after a \akey{check-sat}
command returns \texttt{unsat}.  The unsatisfiable core that is
returned must consist of a list of names of formulas, in the format
prescribed by the SMT-LIB standard.

The result of a solver is considered erroneous if the response to the
\akey{check-sat} command is \texttt{sat}, or if the returned
unsatisfiable core is not well-formed (e.g., contains names of
formulas that have not been asserted before), or if the returned
unsatisfiable core is not, in fact, unsatisfiable.

In order to perform this unsatisfiability check, the organizers will use a
selection of SMT solvers that participate in the \maintrack of this
competition.  For each division, the organizers will use only solvers that have
been sound (i.e., not produced any erroneous result) in the \maintrack for
this division.  The unsatisfiability of a produced unsatisfiable core is
refuted if the number of checking solvers whose result is \texttt{sat} exceeds
the number of checking solvers whose result is \texttt{unsat}.  The time limit
for checking unsatisfiable cores is yet to be determined, but is anticipated to
be around 5 minutes of wall-clock time per solver.

Solvers must respond to each command in the benchmark script with the
answers defined in the SMT-LIB format specification.  In particular,
solvers that respond \texttt{unknown} to the \akey{check-sat} command
must respond with an error to the following \akey{get-unsat-core}
command.

\subsection{\challtrack}
\label{sec:exec:industry-challenge}

\todo{time limit}
The \challtrack will include both incremental and single query benchmarks and
it will follow the same rules as the \inctrack and  \maintrack, respectively
with two exceptions: benchmark selection and the time limit. The track will
run on challenging industrial benchmarks provided by the community. This year,
the \challtrack will include both incremental and single query benchmarks in the
following logics QF\_BV, QF\_ABV and QF\_AUFBV. The complete list of benchmarks
along with instructions on how to access them will be provided on the SMT-COMP
website as soon as the benchmarks are released.

\subsection{\mvaltrack (experimental)}
\label{sec:exec:model}
The \mvaltrack will evaluate the solver's ability to produce models
for satisfiable problems. This track will run on all QF\_BV satisfiable benchmarks
from the \maintrack.

The SMT-LIB language accommodates this functionality by providing the ability to
request a satisfying model after a \akey{check-sat} command returns \texttt{sat}.
The satisfiable core that is returned must consist of definitions specifying all
and only the current user-declared function symbols, in the format prescribed by
the SMT-LIB standard.

The result of a solver is considered erroneous if the response to the
\akey{checks-sat} command is \texttt{unsat}, if the returned model is not
well-formed (e.g. does not provide a definition for all the user-declared function
symbols), or if the returned model does not satisfy the benchmark.

In order to check that the model satisfies the benchmark, the organizers will use
a model validating tool. The organizers will make this tool available to solver
developers before the solver submission deadline.

%%%%%%%%%%%%%%%%%%%%%%%%%%%%%%%%%%%%%%%%%%%%%%%%%%%%%%%%%%%%%%%%%%%%%%%%%%%%%%%%

\section{Benchmarks and Problem Divisions}

\todo{Aina: We do not run non-competitive divisions.}


\header{Benchmark sources.}
%
Benchmarks for each division will be drawn from the SMT-LIB benchmark
library.  The \maintrack will use a subset of all
\emph{non-incremental} benchmarks; the incremental track will use a
subset of all \emph{incremental} benchmarks.  The \ucoretrack
will use unsatisfiable \maintrack benchmarks, modified to use named
top-level assertions.


\header{New benchmarks.}
%
The deadline for submission of new benchmarks is {\bf March~1, 2019}.
The organizers, in collaboration with the SMT-LIB maintainers, will be
checking and curating these until {\bf May~1, 2019}.  The SMT-LIB
maintainers intend to make a new release of the benchmark library
publicly available on or close to this date.

\header{Benchmark demographics.}
%
The set of all SMT-LIB benchmarks in a given division can be naturally
partitioned to sets containing benchmarks that are similar from the user
community perspective.  Such benchmarks could all come from the same
application domain, be generated by the same tool, or have some other
obvious common identity.
%
The organizers try to identify a meaningful partitioning based on the
directory hierarchy in SMT-LIB.  In many cases the hierarchy consists of
the top-level directories each corresponding to a submitter, who has
further imposed a hierarchy on the benchmarks.
%
The organizers believe that the submitters have the best information on
the common identity of their benchmarks and therefore partition each
division based on the bottom-level directory imposed by each submitter.
These partitions are referred to as \emph{families}.

\header{Benchmark selection.} \label{benchmark-selection}
%
The competition will use a large subset of SMT-LIB benchmarks, with some
guarantees on including new benchmarks.  In all tracks the following
process will be used for culling the benchmark pool.
\begin{enumerate}
\item \emph{Remove inappropriate benchmarks.} The competition
  organizers may remove benchmarks that are deemed inappropriate or
  uninteresting for competition, or cut the size of certain benchmark
  families to avoid their over-representation.  SMT-COMP attempts to
  give preference to benchmarks that are ``real-world,'' in the sense
  of coming from or having some intended application outside SMT.
\item \emph{Remove easy benchmarks.} The competition organizers will
  remove all benchmarks that were solved by all solvers (including the
  non-competitive solvers) in the 2018 SMT competition.
\item \emph{Cap the number of instances in a division.} The competition
  organizers will limit the number of benchmarks in a division as
  follows:
  \begin{enumerate}
  \item If a division contains less than 300 instances, all
        the instances will selected.
  \item If a division contains between 300 and 600 instances,
        a selection of 300 instances from the set will be selected.
  \item If a division contains more than 600 instances, 50\% of
        the benchmarks will be selected.
  \end{enumerate}
\item \emph{\ucoretrack{}.} In addition, for the \ucoretrack{}, all
  benchmarks with a single assertion will be removed.
\end{enumerate}
%
The selection process in cases (b) and (c) above will guarantee the
inclusion of new benchmarks by first picking randomly one benchmark from
each new benchmark family.  The rest of the benchmarks will be chosen
randomly from the remaining benchmarks using a uniform distribution.
%
The culling script will be publicly available and will use the same
random seed as the rest of the competition.  The set of benchmarks
selected for the competition will be published when the competition
begins.

\header{Heats.}
%
Since the organizers at this point are unsure how long the set of
benchmarks may take (which will depend also on the number of solvers
submitted), the competition may be run in \emph{heats}.  For each
track and division, the selected benchmarks may be randomly divided
into a number of (possibly unequal-sized) heats.  Heats will be run in
order.  If the organizers determine that there is adequate time, all
heats will be used for the competition.  Otherwise, incomplete heats
will be ignored.

\header{Benchmark scrambling.}
%
Benchmarks will be slightly scrambled before the competition, using a
simple benchmark scrambler.  The benchmark scrambler will be made
publicly available before the competition.

Naturally, solvers must not rely on previously determined identifying
syntactic characteristics of competition benchmarks in testing
satisfiability.  Violation of this rule is considered cheating.

\header{Pseudo-random numbers.}
%
Pseudo-random numbers used, e.g., for the creation of heats or the
scrambling of benchmarks, will be generated using the standard C
library function \texttt{random()}, seeded (using \texttt{srandom()})
with the sum, modulo $2^{30}$, of the integer numbers provided in the
system descriptions (see Section~\ref{sec:entrants}) by all SMT-COMP
entrants other than the organizers'.  Additionally, the integer part
of the opening value of the New York Stock Exchange Composite Index on
the first day the exchange is open on or after the date specified in
the timeline (Section~\ref{sec:important}) will be added to the other
seeding values.  This helps provide transparency, by guaranteeing that
the organizers cannot manipulate the seed in favor of or against any
particular submitted solver.

%%%%%%%%%%%%%%%%%%%%%%%%%%%%%%%%%%%%%%%%%%%%%%%%%%%%%%%%%%%%%%%%%%%%%%%%%%%%%%%%

\section{Scoring \todo{update}}
\label{sec:scoring}

\subsection{Competitive divisions}

Scores will be computed for all solvers and divisions.  However,
winners will be declared only for \emph{competitive} divisions.  A
division in a track is competitive if at least two substantially
different solvers (i.e., solvers from two different teams) were
submitted.  Although the organizers may enter other solvers for
comparison purposes, only solvers that are explicitly submitted by
their authors determine whether a division is competitive, and are
eligible to be designated as winners.

\subsection{Benchmark scoring}
\label{sec:benchmark-scoring}

A solver's \emph{parallel performance} for each benchmark is a quadruple $\langle
e, n, w, c\rangle$, with $e \in \{0, 1\}$ the number of erroneous
results (usually~$e = 0$), $0 \leq n \leq N$ the number of correct
results,\footnote{Here, $N$ is the number of \akey{check-sat} commands
  in the benchmark.  Recall that \maintrack benchmarks have just one
  \akey{check-sat} command; incremental track benchmarks may have
  multiple \akey{check-sat} commands.} $w \in [0,T]$ the (real-valued)
wall-clock time in seconds, and $c \in [0, 4T]$ the (real-valued) CPU
time in seconds, measured across all cores and sub-processes, until
the solver process terminates.

\header{\maintrack.} More specifically, for the \maintrack, we have
%
\begin{itemize}
\item $e=0$, $n=0$ if the solver aborts without a response, or the
  result of the \akey{check-sat} command is \texttt{unknown},
\item $e=0$, $n=1$ if the result of the \akey{check-sat} command is
  \texttt{sat} or \texttt{unsat}, and the result either agrees with
  the benchmark status or the benchmark status is unknown,\footnote{If
    the benchmark status is unknown, we thus treat the solver's answer
    as correct.  Disagreements between different solvers on benchmarks
    with unknown status are governed in
    Section~\ref{sec:division-scoring}.}
\item $e=1$, $n=0$ if the result of the \akey{check-sat} command is
  incorrect.
\end{itemize}
%
Note that a (correct or incorrect) response is taken into
consideration even when the solver process terminates abnormally, or
does not terminate within the time limit.  Solvers should take care
not to accidentally produce output that contains \texttt{sat} or
\texttt{unsat}.

\header{Incremental Track.}
%
An application benchmark may contain multiple \akey{check-sat}
commands.  Solvers may partially solve the benchmark before timing
out.  The benchmark is run by the trace executor, measuring the total
time (summed over all individual commands) taken by the solver to
respond to commands.\footnote{Times measured by StarExec may include
  time spent in the trace executor.  We expect that this time will
  likely be insignificant compared to time spent in the solver, and
  nearly constant across solvers.}  Most time will likely be spent in
response to \akey{check-sat} commands, but \akey{assert}, \akey{push}
or \akey{pop} commands might also entail a reasonable amount of
processing.  For the incremental track, we have
\begin{itemize}
\item $e=1$, $n=0$ if the solver returns an incorrect result for any
  \akey{check-sat} command within the time limit,
\item otherwise, $e=0$, and $n$ is the number of correct results for
  \akey{check-sat} commands returned by the solver before the time
  limit is reached.
\end{itemize}

\header{\ucoretrack.}
%
For the \ucoretrack, we instead have $0 \leq n \leq A$, where~$A$
is the number of named top-level assertions in the benchmark, and
\begin{itemize}
\item $e=0$, $n=0$ if the solver aborts without a response, or the
  result of the \akey{check-sat} command is \texttt{unknown},
\item $e=1$, $n=0$ if the result is erroneous according to
  Section~\ref{sec:exec:unsat-core},
\item otherwise, $e=0$, and $n$ is the reduction in the number of
  formulas, i.e., $n = A$ minus the number of formula names in the
  reported unsatisfiable core.
\end{itemize}

\subsection{Sequential Performance (\maintrack)}
\label{sec:sequential}


\iffalse
SMT-COMP has traditionally emphasized sequential performance (i.e.,
CPU time) over parallel performance (i.e., wall-clock time).  StarExec
measures both, and we intend to recognize both best sequential and
best parallel solvers in all competitive \maintrack divisions.
\fi

The parallel performance, as defined in Section~\ref{sec:benchmark-scoring},
favors parallel solvers, which may utilize all available processor
cores.  To evaluate sequential performance, we derive a
\emph{sequential performance} by imposing a (virtual) CPU time limit equal
to the wall-clock time limit~$T$.  A solver result is taken into
consideration for the sequential performance only if the solver process
terminates within this CPU time limit.  More specifically, for a given
parallel performance $\langle e, n, w, c\rangle$, the corresponding sequential
performance is defined as~$\langle e_S, n_S, c_S\rangle$, where
\begin{itemize}
\item $e_S = 0$ and $n_S = 0$ if $c > T$, $e_S = e$ and $n_S = n$
  otherwise,
\item $c_S = \min\ \{c, T\}$.\footnote{\emph{Rationale:} Under this
  measure, a solver should not benefit from using multiple processor
  cores.  Conceptually, the sequential performance should be (nearly)
  unchanged if the solver was run on a single-core processor, up to a
  time limit of~$T$.}
\end{itemize}

\subsection{Division scoring}
\label{sec:division-scoring}

We compute in total eight different {\em scores} for each division based
on solvers' parallel and sequential performance.  The two performances
are computed separately for {\em all}, {\em satisfiable}, and {\em
unsatisfiable instances}.  In addition we compute the performances for a
\emph{one-minute wonder} track, where each solver is run with a
24-second timeout.
\todo{24-seconds is only a score, not a track}

\header{\maintrack and \challtrack: Removal of Disagreements.}
%
\todo{do we need to update this notion?}
Before division scores are computed for the \maintrack and \challtrack,
benchmarks with unknown status are removed from the competition results if two
(or more) solvers that are sound on benchmarks with known status disagree on
their result.  More specifically, a solver (including a solver that was entered
into the competition by the organizers for comparison purposes) is \emph{sound
on benchmarks with known status} for a division if its parallel performance
(Section~\ref{sec:benchmark-scoring}) is of the form $\langle 0, n, w,
c\rangle$ for each benchmark in the division, i.e., if it did not produce any
erroneous results.  Two solvers \emph{disagree} on a benchmark if one of them
reported \texttt{sat} and the other reported \texttt{unsat}.  Only the
remaining benchmarks are used in the following computation of division scores
(but the organizers may report disagreements for informational purposes).

\header{\inctrack: Removal of Disagreements.}
\todo{}

\header{\challtrack: Removal of Disagreements.}
\todo{}

\medskip

To compute a solver's scores for a division, the solver's individual
benchmark scores for all benchmarks in the division are first
multiplied by a scalar weight that depends on the benchmark's family,
and then summed component-wise.

For a given competition benchmark~$b$, let~$F_b \geq 1$ be the total
number of benchmarks in $b$'s benchmark family that were used in the
competition track to which the division belongs (and not removed
because of disagreements).  We define the weight for benchmark~$b$ as
$\alpha_b = (1 + \log_e F_b) / F_b$.  We define the \emph{normalized weight} for
benchmark~$b$ as $\alpha'_b = \alpha_b / (\sum_{b'} \alpha_{b'})$,
where the sum is over all benchmarks in the division.  Let~$N$ be the
total number of benchmarks in the division.

For \maintrack and \ucoretrack divisions, we will separately
compute the weighted sum of all benchmarks separately based on the
parallel performance and sequential performance.
(Section~\ref{sec:benchmark-scoring})
%
These will be computed as
$$\sum_b \alpha'_b \cdot \langle e_b \cdot N, n_b \cdot N, w_b, c_b\rangle$$
%
where the sum $\sum_b$ ranges over all parallel performances when
computing parallel performance scores, and over all sequential
performances when computing sequential performance scores.
\iffalse
benchmarks in the division to assess parallel performance, and the
weighted sum of all sequential performances (Section~\ref{sec:sequential})
to assess sequential performance.
\fi
For incremental track divisions,
division scores will be based on parallel performance only.\footnote{Since
  incremental track benchmarks may be partially solved, defining a
  useful sequential performance for the incremental track would require
  information not provided by the parallel performance, e.g., detailed timing
  information for each result.}
  \todo{is this only true for the incremental track or also one of the new tracks}

Division scores are compared lexicographically:
\begin{itemize}
\item A weighted sum of parallel performances $\langle e, n, w, c\rangle$ is
  better than $\langle e', n', w', c'\rangle$ iff $e < e'$ or ($e =
  e'$ and $n > n'$) or ($e = e'$ and $n = n'$ and $w < w'$) or ($e =
  e'$ and $n = n'$ and $w = w'$ and $c < c'$).  That is, fewer errors
  takes precedence over more correct solutions, which takes precedence
  over less wall-clock time taken, which takes precedence over less
  CPU time taken.
\item A weighted sum of sequential performances $\langle e_S, n_S,
  c_S\rangle$ is better than $\langle e_S', n_S', c_S'\rangle$ iff
  $e_S < e_S'$ or ($e_S = e_S'$ and $n_S > n_S'$) or ($e_S = e_S'$ and
  $n_S = n_S'$ and $c_S < c_S'$).  That is, fewer errors takes
  precedence over more correct solutions, which takes precedence over
  less CPU time taken.
\end{itemize}
%
We will not make any comparisons between parallel and sequential
performances, as these are intended to measure fundamentally different
performance characteristics.

\subsection{Other Recognitions \todo{Update}}

The organizers will also recognize the following contributions:
%
\begin{itemize}
\item \emph{Best new entrant}. The best performing entrant from a new
  solver implementation team, as measured by the competition-wide
    metric. \todo{}
\item \emph{Benchmarks}. Contributors of new benchmarks.
\end{itemize}
%
The organizers reserve the right to recognize other outstanding
contributions that become apparent in the competition results.

As part of the FLoC Olympic Games, Kurt G{\"o}del medals will be
awarded to:
%
\begin{itemize}
\item the solver that wins the most competitive divisions in the
  incremental track;
\item the solver that wins the most competitive divisions in the
  \ucoretrack;
\item the solver that wins the QF\_BV division (separately for
  parallel and sequential performance) in the \maintrack.
\end{itemize}

%%%%%%%%%%%%%%%%%%%%%%%%%%%%%%%%%%%%%%%%%%%%%%%%%%%%%%%%%%%%%%%%%%%%%%%%%%%%%%%%

\section{Judging}

The organizers reserve the right, with careful deliberation, to remove
a benchmark from the competition results if it is determined that the
benchmark is faulty (e.g., syntactically invalid in a way that affects
some solvers but not others); and to clarify ambiguities in these
rules that are discovered in the course of the competition.  Authors
of solver entrants may appeal to the organizers to request such
decisions.  Organizers that are affiliated with solver entrants will
be recused from these decisions.  The organizers' decisions are final.

%%%%%%%%%%%%%%%%%%%%%%%%%%%%%%%%%%%%%%%%%%%%%%%%%%%%%%%%%%%%%%%%%%%%%%%%%%%%%%%%

\section{Acknowledgments}

SMT-COMP 2019 is organized under the direction of the SMT Steering
Committee. The organizing team is
%
\begin{itemize}
\setlength{\itemsep}{0pt}
\item
		Liana Hadarean~-- Amazon, USA (co-organizer)
\item \href{http://www.inf.usi.ch/postdoc/hyvarinen/}{Antti Hyvarinen}~--
		Universita della Svizzera italiana, Switzerland (co-organizer)
\item
  \href{http://cs.stanford.edu/people/niemetz}{Aina
    Niemetz}~-- Stanford University, USA (co-chair)
\item \href{http://www.cs.man.ac.uk/~regerg/}{Giles Reger}~--
  University of Manchester, UK (co-chair)
\end{itemize}
%
The competition charis are responsible for policy and procedure decisions,
such as these rules, with input from the co-organizers.

Many others have contributed benchmarks, effort, and feedback.  Clark Barrett,
Pascal Fontaine, Aina Niemetz and Mathias Preiner are maintaining the SMT-LIB
benchmark library.
The competition uses the
\href{https://www.starexec.org/}{StarExec} service, which is hosted at
the \href{http://www.cs.uiowa.edu/}{University of Iowa}.  Aaron Stump
is providing essential StarExec support.

\header{Disclosure.}
%
Liana Hadarean was part of the developing team of the SMT solver
CVC4~\cite{cvc4} and is currently not associated with any group
creating or submitting solvers.
Antti Hyvarinen is part of the developing team of the SMT solver
OpenSMT~\cite{opensmt2}.
Aina Niemetz is part of the developing teams of the SMT solvers
Boolector~\cite{boolector}
and CVC4~\cite{cvc4}.
Giles Reger is associated with the group
producing the Vampire system~\cite{vampire}.

%%%%%%%%%%%%%%%%%%%%%%%%%%%%%%%%%%%%%%%%%%%%%%%%%%%%%%%%%%%%%%%%%%%%%%%%%%%%%%%%

\bibliographystyle{plain}
\bibliography{biblio}

%%%%%%%%%%%%%%%%%%%%%%%%%%%%%%%%%%%%%%%%%%%%%%%%%%%%%%%%%%%%%%%%%%%%%%%%%%%%%%%%

\end{document}

%% Local Variables:
%% mode: latex
%% mode: flyspell
%% ispell-local-dictionary: "american"
%% LocalWords: arity Heizmann logics Reger satisfiability SMT StarExec Tjark
%% End:
