\documentclass[12pt]{article}
\usepackage{times}
\usepackage{fullpage}
\usepackage{url}
\usepackage{epsf}

\newcommand{\akey}[1]{\textbf{#1}}

\begin{document}

\date{\small This version revised \the\year-\the\month-\the\day}

\title{Satisfiability Modulo Theories Competition (SMT-COMP) 2012: Rules and 
Procedures}


% [morgan] do our own layout of authors; the four-author layout spacing
% was screwed up...
\def\doauthor#1{{%
  \hsize.5\hsize \advance\hsize by-1cm %
  \def\\{\hss\egroup\hbox to\hsize\bgroup\strut\hss}%
  \vbox{\hbox to\hsize\bgroup\strut\hss#1\hss\egroup}}}%

\def\header#1{\medskip\noindent\textbf{#1}}

\author{
Roberto Bruttomesso\\
Dept. of Computer Science\\
University of Milan, Milan (Italy)
\and
David R. Cok \\
GrammaTech, Inc. \\
Ithaca, NY (USA) \\
\and
Alberto Griggio\\
ES Division\\
FBK, Trento (Italy)
}

\maketitle

\def\eg{\textit{e.g.}}
\def\ie{\textit{i.e.}}

\noindent Comments on this document should be emailed to the smtcomp mailing
list or, if necessary, directly to the organizers.

\section{Introduction}
\label{sec:intro}

The annual Satisfiability Modulo Theories Competition~(SMT-COMP) is
held to spur advances in SMT solver implementations on benchmark
formulas of practical interest.  Public competitions are a well-known
means of stimulating advancement in software tools.  For example, in
automated reasoning, the CASC and SAT competitions for first-order and
propositional reasoning tools, respectively, have spurred significant
innovation in their fields~\cite{PSS02,leberre+03}.  More information
on the history and motivation for SMT-COMP can be found at the
SMT-COMP web site, \url{www.smtcomp.org}, and in reports on previous
competitions~(\cite{SMTCOMP-2008,BDOS08,SMTCOMP-FMSD,SMTCOMP-JAR}).
SMT-COMP 2012 is affiliated with the SMT workshop (\url {http://smt2012.loria.fr/}) at the 6th International
Joint Conference on Automated Reasoning~(IJCAR) (\url{http://ijcar.cs.manchester.ac.uk/}).

Accordingly, researchers are highly encouraged to submit both new benchmarks
and new or improved solvers to raise the level of competition and advance
the state-of-the-art in automated SMT problem solving.

Note that SMTCOMP 2012 has several tracks: a main track, a parallel track, an
application track, an unsat-core track, and a proof-generation track.
Within a track there are one or more divisions, where each division
uses benchmarks from a specific SMT-LIB logic (or group of logics).

The rest of this document, updated from the last year's
version\footnote{Earlier versions of this document include contributions from
Clark Barrett, Albert Oliveras, Aaron Stump, and Morgan Deters.},
describes the rules and competition procedures for SMT-COMP~2012.
The principal changes from last year's version are the following:
\begin{itemize}

\item In the main and parallel competition, we are concentrating on just a few of the benchmark divisions this year.
  Some divisions, such as QF\_LIA, are subsumed into more expressive
  logics (QF\_UFLIA for QF\_LIA); others have received only light interest in past competitions. Our goal is to
  focus the competition on divisions of particular interest to
  applications. Solvers requesting to be run against other benchmark divisions are
  welcome and will be run in exhibition mode:
  results will be displayed and publicly reported.

\item We are retaining and accenting last year's ``application track" and
  encourage submission of benchmarks for this track and competition against these
  benchmarks.

\item The penalty for producing an incorrect result is increased. Competitors
  producing all correct results are ranked above those producing
  incorrect results (a ``soft" disqualification for unsound tools).

\item We are adding an ``unsat core" track.

\item We will have an exhibition track of solvers that produce proofs.

\item The competition will be run with the SMT-Exec service used last year.
  The organizers had expected to use a new StarExec service, replacing
  the previous SMT-Exec service. However, at this writing the new service
  is not yet available.

\end{itemize}

Additionally, non-competitive divisions will not be exhibited as
part of the competition: at least two competitors must be entered
into a division for it to run as part of the competition.
If necessary, and based on final submissions, the organizers may elect to
combine competition divisions to make them competitive (such decisions will
be made with input from the community).

As in SMT-COMP 2011, the 2012 version will incorporate a small number of
random ``fuzzer-generated" instances (see below) to help promote
attention to solver correctness.

It is important for competitors to track discussions on the SMT-COMP mailing
list, as clarifications and any updates to these rules will be posted there.

\section{Entrants}
\label{sec:entrants}

\header{Solver format.} %
An entrant to SMT-COMP is an SMT solver
submitted using the SMT-Exec\footnote{The organizers may switch to StarExec, 
though Star-Exec is still under development as of 12 May 2012.} service.  
The execution service enables members of the
SMT research community to run solvers on jobs consisting of benchmarks
from the SMT-LIB benchmark library.  Jobs are run on a shared computer
cluster.  The execution service is provided free of
charge, but it does require a minimal registration, which verifies an
email address and prevents misuse of the service.  Registered users
may then upload their own solvers to run, or may run public solvers
already uploaded to the service.  The service provides a variety of tabular
and graphical displays of results of solver executions, for
comparison. For the main (sequential-solver)
% rb 23/3: I guess also application track runs on 
% the same clusters as sequential
and application tracks 
of SMT-COMP 2012, the service will be configured so that jobs are run on a 64-bit,
uniprocessor Linux kernel.  For the parallel-solver track, StarExec
will be configured so that jobs are run on a 64-bit, multiprocessor
Linux kernel.

For participation in SMT-COMP, a solver must be
uploaded as a ``competition'' solver via the service's upload
mechanism, or, alternatively, a previously-uploaded solver may be
marked as a ``competition'' solver.  In either case, \emph{uploads
must be marked for competition before the deadline}; uploading a solver
to the service is not sufficient for competition entry if it is not marked
as being a competition entrant.  The md5 checksums of competition
submissions will be public immediately after the deadline for competition
entry has passed to ensure transparency, and the submissions themselves
will be made public after the competition.  Source code need not be
provided.  However, in order to encourage sharing of source code, extra
recognition will be given to solvers providing source code distributions
including recognition for the top such solver in each division.  Instructions
for uploading solvers and machine specifications will be posted as soon
as they are available.

\header{System description.} %
As part of their submission,
SMT-COMP entrants must also include a short (1--2 pages) description of
the system.  This should include a list of all authors of the system
and their present institutional affiliations.  The programming
language(s) and basic SMT solving approach employed should be
described (\eg, lazy integration of a Nelson-Oppen combination with
SAT, translation to SAT, etc.).  System descriptions are encouraged to
include a URL for a web site for the submitted tool, but this is
optional.  System descriptions must also include a 32-bit unsigned
integer.  These numbers, collected from all submissions, are used
to seed the pseudo-random benchmark selection algorithm, as well
as the benchmark scrambler.

The upload system will ask for the system description and
the random seed when a solver is marked for competition, so
their inclusion in the uploaded archive itself is optional.

\header{Other systems.} %
As in previous years, due to limitations on
computational resources, the organizers reserve the right not to
accept multiple versions of the same solver (defined as sharing 50\%
or more of its source code).  The organizers reserve the right to
submit their own systems, or other systems of interest, to the
competition.

\header{Wrapper tools.} %
A \emph{wrapper tool} is defined as any tool
which calls an SMT solver not written by the author of the wrapper
tool.  The other solver is called the \emph{wrapped tool}.  There are
several rules governing wrapper tools that wrap tools that have been or could be submitted as independent entrants.\footnote{A wrapper
for a wrapped tool that does not accept SMT-LIBv2 format is considered to simply be an SMT version of the underlying tool.}
For the purposes of these rules,
multiple versions of a wrapped tool are considered different tools.
The goal of these rules is to require wrapper tools to outperform the
tools they wrap (since otherwise, there is no apparent quantitative
way to argue that the wrapper tool improves upon the wrapped tool).

\begin{itemize}
\item The name of the wrapper tool must end with ``+name'', where name
is the name of the wrapped tool (\eg\ ``Flash+CVC3'' for a tool
wrapping CVC3).

\item If the wrapped tool is from last year's SMT-COMP or earlier, then for
each division entered by the wrapper tool, if the wrapper tool does not place ahead, according to the scoring
rules below, of last year's winner in that division, it will be disqualified
from that division (but not necessarily from the whole competition).

\item If the wrapped tool was released after last year's SMT-COMP,
then the wrapper tool can be entered only if

\begin{itemize}
\item Permission has been given by the author of the wrapped tool

\item The wrapped tool is submitted and entered in every division
      in which the wrapper tool is entered.
\end{itemize}

For each division entered by the wrapper tool, if the wrapper tool
does not place ahead of the wrapped tool in that division, it will be
disqualified from that division.
\end{itemize}

\header{Attendance.} %
As with previous SMT-COMPs, submitters of an SMT-COMP entrant need not 
be physically present at the competition to
participate or win.

\subsection*{Deadlines} %

\paragraph{Main competition track.} %
SMT-COMP entries must be submitted via SMTExec by 7pm, Eastern U.S.
time, June~15, 2012.  At that time the service will be
closed to the public to prepare for the competition, with the
exception that resubmissions of existing entries will be accepted
until 7pm, Eastern U.S. time, June~18, 2012.  We strongly encourage
participants to use this weekend grace period \emph{only} for the
purpose of fixing any bugs that may be discovered and not for adding
new features as there will be no opportunity to do extensive testing
using StarExec after the original deadline on June~15.

The versions that are present on the execution service at the conclusion of the
grace period will be the ones used for the competition, and versions
submitted after this time will not be used.  The organizers reserve
the right to start the competition itself at any time after the open
of the New York Stock Exchange on the morning of June~20.  See
Section~\ref{sec:timeline} below for a full timeline.

\paragraph{Application track.} %
The same deadlines and procedure for submitting to the main track will be used
also for the application track. Submissions to both the application track and
the main competition are independent: participants must submit explicitly to
both events to participate in both, and they may submit different (or
differently configured) solvers to each.  Benchmarks will be scrambled also
for this division, using the same scrambler and seed as the main track.
Entrants should still include a system description, as for the regular
competition.

\paragraph{Parallel solver track.} %
Solvers employing concurrency are invited to participate in a
demonstration that will be run in conjunction with the regular competition.
Parallel solvers must be submitted by the same June~15th deadline,
with the same grace period for resubmission as in the regular
competition.  Submissions to both the parallel solver track
and the main competition are independent: participants must submit
explicitly to both events to participate in both, and they may submit
different (or differently configured) solvers to each.  Benchmarks may
or may not be scrambled for this division.  Entrants should still
include a system description, as for the regular competition.

\medskip
\noindent
Note that the SMT-COMP organizers may include
other (``historical'' or otherwise relevant) solvers in all the competition tracks for
demonstration and comparison.  The organizers reserve the right to
include or exclude such solvers, and to make simple modifications
to historical sequential solvers to (na\"ively) take advantage of
multiple processors 
or to connect them with a parser for the SMT-LIB version 2.0 format
(\eg\ by implementing a wrapper tool).

\section{Execution of Solvers}
\label{sec:exec}

Solvers will be publicly evaluated in the following tracks, listed here and 
described in detail below. In exhibition divisions of these tracks, solvers are evaluated and results posted, but no awards are announced. The other divisions are competition divisions; if there are sufficient entrants, a winner will be announced.
\begin{itemize}
\item a main track: sequential execution evaluated separately on benchmarks from each of several different logics (some competitive and some exhibition divisions)
\item a parallel track: parallel execution evaluated on the same benchmarks and the same divisions as the main track (some competitive and some exhibition divisions)
\item application track: evaluation on command scripts (all divisions with sufficient benchmarks and entrants are competitive)
\item unsat-core track: a competition among solvers capable of computing unsat cores
\item proof-generation exhibition: an exhibition track for solvers capable of generating proofs (this track is tentative, given that the organizers must work with the SMT-Exec service)
\end{itemize}

\subsection{Logistics}

\header{Dates of competition.}
%
We anticipate that the bulk of the main track of the competition will take place during
the course of IJCAR 2012, from June~26 to the 29th.  Results will be
announced in a special session of IJCAR, on the last day of the
conference, as well as at the SMT workshop and on the SMT-COMP web site.  Intermediate results
will be regularly posted to the SMT-COMP website as the competition
runs.

If there are enough competitors to run the parallel track, we intend
to run it before or after the main competition (depending on the
estimated duration of the track). We may choose to use an NYSE seed from a different date
to accommodate an early start.

The organizers will prioritize the running of the competition tracks, and may shift the
time period or order of the competition or exhibition tracks in order to complete SMT-COMP
in the course of the IJCAR conference.

\header{Input and Output.}
%
Participating solvers must read a single benchmark script (defined
below, not part of the 2.0 standard), presented on its standard input
channel. The script is in the concrete syntax of the SMT-LIB format,
version 2.0.  A benchmark script is essentially just the translation
of a benchmark from the version 1.2 specification.  In more detail, a
\textbf{benchmark script} is just a script where:

\begin{enumerate}
\item The (single) $\akey{set-logic}$ command setting the benchmark's
logic is the first command after any $\akey{set-option}$ commands described below.
% rb 25/3: added
% \item A (single) \akey{set-option :print-success false} command, sent just
%       after $\akey{set-logic}$. This is to
%       avoid the output ``\texttt{success}'' to interfere with the
%       answer of the solver to the benchmark. 
\item The $\akey{exit}$ command is the last command.
\item For tracks other than the application track, there is exactly one $\akey{check-sat}$ command,
following possibly several $\akey{assert}$ commands.
\item For the application track, there are one or more \akey{check-sat} commands, 
  each preceded by one or more \akey{assert} commands 
  and zero or more \akey{push 1} commands, 
  and followed by zero or more \akey{pop 1} commands.
\item Scripts for the application track will have an initial $\akey{set-option :print-success true}$ command.
\item There is at most one $\akey{set-info}$ command for \texttt{status}.
%%%%%%%%%%%%%%%%%%% TODO: Need to define the set-info command.
\item The formulas in the script belong to the benchmark's logic, with
any free symbols declared in the script.
\item Extra symbols are declared exactly once before any
  use, using $\akey{declare-sort}$
  %, $\akey{define-sort}$,
  %$\akey{declare-fun}$, or $\akey{define-fun}$.  
  or $\akey{declare-fun}$.
  They must be part of the allowed signature expansion for the logic.
  Moreover, all sorts declared with a $\akey{declare-sort}$ command must have zero arity.
\item In the unsat-core competition, the $\akey{set-logic}$ command is preceded by a \\
$\akey{set-option :produce-unsat-cores true}$ command and the 
$\akey{check-sat}$ command is followed by a $\akey{get-unsat-core}$ command.
Also, some or all of the $\akey{assert}$ commands will assert named formula
(for example, \akey{assert (! P :named F1)}). The $\akey{get-unsat-core}$
command must return a parenthesized list of formula names, as specified by the 
SMTLIBv2 standard.
\item In the proof generation competition, the $\akey{set-logic}$ command is preceded by a \\
$\akey{set-option :produce-proofs true}$ command and the 
$\akey{check-sat}$ command is followed by a $\akey{get-proof}$ command.
The $\akey{get-proof}$ command must return a proof in a solver-dependent format made known to the organizers.
\item No other commands besides the ones just mentioned may be used.

\end{enumerate}

  
\noindent The SMT-LIB format specification is publicly
available from the ``Documents'' section of the SMT-LIB
website~\cite{SMT-LIB}.  Solvers will be given formulas just from the
Problem Divisions indicated during their submission to SMT-Exec.
Example benchmark scripts for several Problem Divisions are reported in the Appendix.
Note that they are provided for illustrative purposes only: 
please refer to the SMT-LIB format specification 
and the above definition of benchmark script for the official specification 
of the input format for SMT-COMP.

\subsection{Main track and parallel track}
\label{sec:exec:main}

The main track competition will consist of selected benchmarks in each of the
logic divisions given below.
Each benchmark script will be presented to the standard input of the solver.
Each SMT-COMP entrant is then expected to attempt to report on its
standard output channel whether the formula is satisfiable
(``\texttt{sat}'', in lowercase, without the quotation marks) or unsatisfiable
(``\texttt{unsat}'').  An entrant may also report ``\texttt{unknown}''
to indicate that it cannot determine satisfiability of the formula.
For more detailed information on the output format, see the
description on the StarExec ``Upload a Solver'' page.  %%%%%%% TODO: CHeck this and all referecnes to StarExec

Solvers that register in non-competition logic divisions (cf. section \ref{sec:theories})
will be run in an exhibition mode: results will be generated and publicly
posted, but no awards will be announced.

\header{Timeouts.} %
Each SMT-COMP solver will be executed on an
unloaded competition machine for each given formula, up to a fixed
time limit.  The time limit is yet to be determined, but it is
anticipated to be 20 minutes, as it was in 2011.\footnote{The time limit may be adjusted once we understand the resources and capabilities of the StarExec service.} 
Solvers that take more than this
time limit will be killed.  Solvers are allowed to spawn other
processes.  These will be killed at approximately the same time as the
first started process, using the TreeLimitedRun script, developed for
the CASC competition and available on the SMT-COMP web page.  A
timeout scores the same as if the output is ``\texttt{unknown}''.
%%%%%%%%%%% TODO: Clarify time limit

\header{Aborts and unparsable output.} %
Solvers which exit before the time
limit without reporting a result (\ie\ due to exhausting memory, crashing,
or producing output other than \texttt{sat}, \texttt{unsat}, or
\texttt{unknown})
will be considered to have aborted. 
An abort scores the same as if the output is ``\texttt{unknown}''. 
% rb 25/3: added
Also, as a further measure to prevent misjudgments of solvers,
any  ``\texttt{success}'' outputs will be 
ignored.\footnote{
Note that SMT-LIBv2 requires to produce a ``\texttt{success}'' answer
after each \akey{set-logic}, \akey{declare-sort}, \akey{declare-fun} and
\akey{assert} command (among others), unless the option
\akey{:print-success} is set to false; ignoring the
\texttt{success} outputs therefore allows for submitting fully-compliant
solvers without the need of a wrapper script, while still allowing entrants
of previous competitions to run without changes.}

\header{Persistent state.} %
Solvers are allowed to create and write to
files and directories during the course of an execution, but they are
not allowed to read such files back during later executions.  Any
files written should be put in the directory in which the tool is
started, or in a subdirectory.

\subsection{Application track}
\label{sec:exec:application}

The application track evaluates SMT solvers when interacting
with an external verification framework, \eg, a model
checker. This interaction, ideally, happens by means of an online
communication between the model checker and the solver: the model
checker repeatedly sends queries to the SMT solver, which in turn
answers either \texttt{sat} or \texttt{unsat}.  In this interaction an SMT-solver is
required to accept queries incrementally via its standard input channel.

In order to facilitate the evaluation of the solvers in this track, we
will set up a ``simulation'' of the aforementioned interaction, as was done in 2011. In
particular each benchmark in the application track represents a realistic
communication trace, containing multiple \akey{check-sat} commands (possibly
with corresponding \akey{push 1}/\akey{pop 1} commands), which
is parsed by a {\em trace executor}. The trace executor serves the following purposes:
\begin{itemize}
\item it simulates the online interaction by sending single queries to the SMT solver
      (through stdin);
\item it prevents ``look-ahead'' behaviours of SMT solvers;
\item it records time and answers for each call, possibly aborting the execution
      in case of a wrong answer;
\item it guarantees a fair execution for all solvers by abstracting from any possible
      crash, misbehaviour, etc. that may happen on the model checker side.
\end{itemize}

\header{Input and Output.}
Participating solvers will be connected to a trace executor 
which will incrementally send commands to the standard input channel of the solver
and read responses from the standard output channel of the solver.
The commands will be taken from an \textbf{incremental benchmark script},
which is an SMT-LIB 2.0 script which satisfies the rules for an application script given above.
Note also that the trace executor will send a single 
\akey{set-option :print-success true} command to the solver before 
sending commands from the incremental benchmark script.

\medskip
Solvers must respond immediately to the commands sent by the trace executor, 
with the answers defined in the SMT-LIB 2.0 format specification, that is,
with a \texttt{success} answer for 
\akey{set-option}, 
\akey{set-logic}, 
\akey{declare-sort}, 
\akey{declare-fun}, 
\akey{assert}, 
\akey{push 1}, 
and \akey{pop 1} 
commands,
 with a \texttt{sat}, \texttt{unsat}, or \texttt{unknown} 
for \akey{check-sat} commands, and with the defined responses for \akey{get-unsat-core} and \akey{get-proof} commands.

\header{Timeouts.}
A time limit is set for the whole execution of each application
benchmark (consisting of multiple queries).  We anticipate the
timeout to be around 30~minutes (as it was in 2011).\footnote{The timeout may be adjusted once we have experience with the abilities and resources of the StarExec service.}


\subsection{Unsat-core and proof-generation tracks}

Applications such as software verification and model checking are enhanced by having
proofs and unsatisfiable cores available from SMT solvers. Accordingly we are 
encouraging solvers to add such capabilities by recognizing them in SMTCOMP 2012.

The unsat-core track will evaluate solvers' capability to generate unsatisfiable cores for
problems that are known to be unsatisfiable. Solvers will be measured by the smallness of
the unsat core they return. The SMT-LIBv2 language accommodates this functionality by 
providing two features: the ability to name top-level (asserted) formula and the ability to
request an unsat-core after a check-sat command returns \texttt{unsat}. The unsat-core that
is returned consists of a list of names of formula. In the competition we will check that the
returned unsat-core is well-defined and is still unsatisfiable.

Similarly, the competition will feature an exhibition track of proof-generating solvers. This is
simply an exhibition track because there is as yet no SMT-LIB standard way to express proofs and no
means of checking the proofs. We hope
that encouraging the capability in solvers will also encourage a common proof format. The exhibition
will count the number of benchmarks for which a solver successfully generates a proof.
Solvers submitted against this track must be accompanied by a description of the format of the 
generated proofs. 

The organizers are still reviewing which logic divisions will be supported for the unsat-core and proof generating tracks; 
those divisions will be announced later.
Input from teams considering submitting unsat-core or proof generating solvers is welcome. In addition, because
the competition has had to revert to the SMT-Exec, the organizers may simply make the proof genreation demonstration 
informational or omit it.
%% TODO

\section{Benchmarks and Problem Divisions}
\label{sec:theories}


The competition divisions for SMT-COMP 2012 are planned to be the following
SMT-LIB \emph{logics}.  These logics are specified in SMT-LIB format on the
SMT-LIB web page.  However, the organizers reserve the right to add
(remove) divisions if (not) enough benchmarks and solvers exist for a
particular division.

Note that this year we are reducing the number of competitive benchmark divisions.
Our reasons are to concentrate the competition and to focus development and attention
on benchmarks that are more challenging and relevant to applications.

In addition, the organizers reserve the right to include in the benchmark population benchmarks
from subsumed logics. For example, QF\_UFLIA may include some samples from QF\_LIA, and QF\_UFLRA may include some from QF\_LRA.
We will not use benchmarks from QF\_UF in other divisions.

\begin{itemize}

\item QF\_BV: fixed-width bitvectors. This logic is important to ``bit-blasting'' model checking of software.
\item QF\_AUFBV: arrays, fixed-width bitvectors and uninterpreted
functions. This logic is also key to software model-checking, but adds arrays (which can be used for memory models) and uninterpreted functions.

\item QF\_UFLIA: uninterpreted functions and linear integer arithmetic. This division evaluates reasoning about integers.
\item QF\_UFLRA: uninterpreted functions and linear real arithmetic. This division evaluates reasoning about real numbers.

\item QF\_IDL: quantifier-free formulas to be
  tested for satisfiability modulo a background theory of integer
  arithmetic.  The syntax of atomic formulas is restricted to
  difference logic, i.e. x - y op c, where op is either equality or
  inequality and c is an integer constant. This division is included because it was highly popular in 2011.

\item AUFLIA$+p$: (quantified) arrays, uninterpreted functions and 
linear integer arithmetic, patterns included. This and the subsequent divisions evaluate solvers' abilities to handle quantified formula.
\item AUFLIA$-p$: (quantified) arrays, uninterpreted functions and 
linear integer arithmetic, patterns not included.
\item AUFNIRA: (quantified) arrays, uninterpreted functions and 
mixed nonlinear integer and real arithmetic

\end{itemize}

\subsection{Main and parallel tracks}

\header{Benchmark sources.} %
Benchmark formulas for these divisions
will be drawn from the SMT-LIB library.  Any benchmarks added to
SMT-LIB by the April 15th release (see the timeline in
Section~\ref{sec:timeline}) will be considered eligible.  SMT-COMP
attempts to give preference to benchmarks that are ``real-world,'' in
the sense of coming from or having some intended application outside
SMT.

\header{Benchmark availability.} % A first release of the competition
Benchmarks will be made available by April 15, 2012.
% A second and almost final release will be available on June 1st. 
No additional
benchmarks will be added after this date, but benchmarks may be
modified or removed to fix possible bugs or other issues, or to adjust their difficulty score (see below). 
The final release that will be used for the competition will be posted on June
1. The set of selected benchmarks will be published when the
competition begins.

\header{Benchmark demographics.} %
In SMT-LIB, benchmarks are organized according to \emph{families}.  A benchmark
family contains problems that are similar in some significant way.  Typically
they come from the same source or application, or are all output by the same
tool.  \emph{Each top-level subdirectory within a division represents a distinct
family.}  
%

Each benchmark in SMT-LIB also has a \emph{category}.  There are four possible
categories:
%
\begin{itemize}
\item \emph{check.} These benchmarks are hand-crafted to test whether
  solvers support specific features of each division.  In particular,
  there are checks for integer completeness (\ie\ benchmarks that are
  satisfiable under the reals but not under the integers) and big
  number support (\ie\ benchmarks that are likely to fail if integers
  cannot be represented beyond some maximum value, such as
  $2^{31}-1$).

  \noindent%\textbf{New for 2010:} 
  Using the same random seed as
  for benchmark selection and scrambling, 5 ``check'' benchmarks will
  be randomly generated using Robert Brummayer's SMT fuzzing
  tool~\cite{brummayer+09}.  The (exact version of this) tool will be
  publicly available from the SMT-COMP~2012 web site before the
  competition.  Since these benchmarks are generated after the random
  seed is fixed, they cannot be known in advance to any competitors,
  including the organizers.  The rationale for including these
  benchmarks is to try to take a step towards stronger certification
  that solvers are correct.  In future years, we envision SMT-COMP
  requiring that solvers pass some kind of pre-qualifying round based
  on SMT fuzzing.  The inclusion of these fuzzing benchmarks that are
  not known in advance is intended to encourage (without requiring)
  solver implementors to test their solvers using fuzzing or similar
  bug-discovery techniques.
\item \emph{industrial.} These benchmarks come from some real application
      and are produced by tools such as bounded model checkers, static analyzers, extended
      static checkers, etc.
\item \emph{random.} These benchmarks are randomly generated.
\item \emph{crafted.} This category is for all other benchmarks.  Usually,
  benchmarks in this category are designed to be particularly difficult or to
  test a specific feature of the logic.
\end{itemize}

\header{Benchmark selection.} %
Before the selection process, each benchmark will be assigned a
\emph{difficulty:} a number between 0.0 and 5.0 inclusive, calculated as in 2011.  The
difficulty for a particular benchmark will be assigned by running SMT
solvers from previous competitions that finished in good
standing and using the formula:

\[\mathrm{difficulty} = {5\cdot\ln\!\left(1+A^2\right) \over \ln\!\left(1+30^2\right)}\]

\noindent
where $A$ is the average time for the solvers to correctly solve the
instance (in minutes).  This computation of difficulties
replaces a simpler formulation in earlier SMT-COMPs that didn't take
into account the time solvers take.  This calculation of difficulty
recognizes that problems requiring more time by many solvers are
more difficult problems.  The logarithm is used to
mark a larger change in difficulty (given a corresponding increase in
solver average time) at smaller time scales than at higher ones (if $A=1$,
difficulty is 0.5; at $A=1.7$, difficulty is 1.0; but a difficulty of 2.0
requires that $A=4$); the square is used to flatten out this curve sightly
at the low end.  
%%Figure~\ref{difficultyplot} shows this curve.
%%
%%\begin{figure}
%%  \centerline{\epsffile{difficultyplot.eps}}
%%  \caption{The shape of the difficulty assignment curve.}
%%  \label{difficultyplot}
%%\end{figure}

When 5 or more solvers are used for the
calculation, the maximum and the minimum times are dropped from the calculation of the average.  Solvers giving an
incorrect answer are not counted in the average; solvers crashing,
timing out, or giving an unknown result are considered as taking 30
minutes (which, with the above formula, pulls the difficulty
toward~5.0).  If there are available solvers, but no average is
defined under these rules, the difficulty shall be 5.0.  For new
divisions, where there are no available solvers to compute the
difficulty, the difficulty will be computed using whatever means are
available to the organizers for that purpose.

The following scheme will be used to choose competition benchmarks
within each division.  Unknown-status benchmarks from SMT-LIB are
considered ineligible for competition and are not used.
%
The selection
is implemented by our benchmark selection tool, source for which will
be available at \url{www.smtcomp.org}.
%%% TODO: Make sure this tool is available.

\begin{enumerate}

\item \textbf{Check benchmarks included.} %
  All benchmarks in category \emph{check} are included.

\item \textbf{Retire very easy benchmarks.} %
  The most difficult~300 non-check non-unknown benchmarks in each
  division are always included, together with all benchmarks on which
  at least one 2011 solver required more than 5~seconds.  \emph{This
    is intended to have the effect of retiring ``very easy''
    benchmarks that were solved by every 2011 solver in less
    than 5~seconds, \emph{unless} doing so reduces the pool of
    benchmarks for the division to less than~300.}

\item \textbf{Retire inappropriate benchmarks.} %
  The competition organizers will remove from the eligibility pool
  certain SMT-LIB benchmarks that are inappropriate or uninteresting
  for competition, or cut the size of certain benchmark families to
  avoid their over-representation.

\item\label{step:pool} \textbf{Division selection pools created.} %
  For non-\emph{check} benchmarks, selection pools are created.
  For benchmark families with $\leq200$ eligible,
  non-\emph{check} benchmarks, all are added to this pool;
  otherwise, 200 such benchmarks are added to the pool with the
  following distribution:
%
  \begin{itemize}
  \item 20 with solution \textbf{sat} and difficulty on $\left[0,1\right]$
  \item 20 with solution \textbf{sat} and difficulty on $\left(1,2\right]$
  \item 20 with solution \textbf{sat} and difficulty on $\left(2,3\right]$
  \item 20 with solution \textbf{sat} and difficulty on $\left(3,4\right]$
  \item 20 with solution \textbf{sat} and difficulty on $\left(4,5\right]$
  \item 20 with solution \textbf{unsat} and difficulty on $\left[0,1\right]$
  \item 20 with solution \textbf{unsat} and difficulty on $\left(1,2\right]$
  \item 20 with solution \textbf{unsat} and difficulty on $\left(2,3\right]$
  \item 20 with solution \textbf{unsat} and difficulty on $\left(3,4\right]$
  \item 20 with solution \textbf{unsat} and difficulty on $\left(4,5\right]$
  \end{itemize}
%
  If 20 are not available in one of these subdivisions, all that are
  available are added, and remaining slots are reallocated to the
  others.  This process is iterated so that it is guaranteed that 200
  benchmarks from the benchmark family are in the selection pool, in
  equal numbers from each subdivision, so far as possible.
  (In cases where, \eg, there are only two available slots and
  they can be allocated to one of three subdivisions, they are allocated
  randomly but are guaranteed to be allocated to \emph{distinct}
  subdivisions.)

\item\textbf{Category slot allocation.} %
  Next, 200 slots are allocated for the division as follows:\footnote{The
  number ``200'' is a guideline, and is expected to be used.  However, the
  SMT-COMP organizers reserve the right to reduce this to an appropriate
  value to ensure a timely end to the competition, and may do so on a
  per-division basis.  The category allotments, etc., will remain the same
  proportion of the total.}
%
  \begin{itemize}
  \item 170 from category \emph{industrial}
  \item 20 from category \emph{crafted}
  \item 10 from category \emph{random}
  \end{itemize}
%
  If there are fewer than 20 (respectively, 10) \emph{crafted} or
  \emph{random} benchmarks in the division pool, more
  \emph{industrial} slots are allocated to make 200 total for the
  division.  If there are too few \emph{industrial} benchmarks in the
  division pool, more \emph{crafted} slots are allocated to make 200
  total for the division.  (In no division are there not enough of
  both industrial and crafted benchmarks.)

\item\textbf{Category subdivision slot allocation.} %
  For each category, given that it has $n$ slots allocated to it,
  the slot allocation is further subdivided as follows:
%
  \begin{itemize}
  \item $\left\lfloor n\over10\right\rfloor$ slots for solution \textbf{sat} with difficulty on $\left[0,1\right]$
  \item $\left\lfloor n\over10\right\rfloor$ slots for solution \textbf{sat} with difficulty on $\left(1,2\right]$
  \item $\left\lfloor n\over10\right\rfloor$ slots for solution \textbf{sat} with difficulty on $\left(2,3\right]$
  \item $\left\lfloor n\over10\right\rfloor$ slots for solution \textbf{sat} with difficulty on $\left(3,4\right]$
  \item $\left\lfloor n\over10\right\rfloor$ slots for solution \textbf{sat} with difficulty on $\left(4,5\right]$
  \item $\left\lfloor n\over10\right\rfloor$ slots for solution \textbf{unsat} with difficulty on $\left[0,1\right]$
  \item $\left\lfloor n\over10\right\rfloor$ slots for solution \textbf{unsat} with difficulty on $\left(1,2\right]$
  \item $\left\lfloor n\over10\right\rfloor$ slots for solution \textbf{unsat} with difficulty on $\left(2,3\right]$
  \item $\left\lfloor n\over10\right\rfloor$ slots for solution \textbf{unsat} with difficulty on $\left(3,4\right]$
  \item $\left\lfloor n\over10\right\rfloor$ slots for solution \textbf{unsat} with difficulty on $\left(4,5\right]$
  \end{itemize}
%
  Remaining slots are allocated randomly to distinct subdivisions.
  If there aren't enough benchmarks in the pool meeting one or more of
  the above subdivision requirements for the category, the subdivision
  allocation is reduced to the number available in the pool that meet
  the requirements.  To make up the full category allotment, remaining
  slots are allocated equally to subdivisions with enough benchmarks
  in the pool meeting their requirements.  This process ensures that
  all slots can be filled with benchmarks from the pool.

\item\textbf{Benchmark selection.} %
  Benchmarks from the pool are assigned randomly to slots.
\end{enumerate}
%
In the end, up to 200 non-\emph{check} benchmarks per division are
included in the competition, together with all the \emph{check}
benchmarks.  Some divisions may have fewer than 200 non-\emph{check}
benchmarks, in which case all of them are included using this selection
scheme.

Once the StarExec service is evaluated, and if the organizers decide that
the service has sufficient resources, the target number of benchmarks for 
a division may be increased above 200, with all subpopulations being enlarged
proportionately, insofar as there are sufficient benchmarks. The target number
of benchmarks will not be raised if that requires reducing the timeout limits
below the values used in 2011.

The main purpose of the algorithm above is to have a balanced and complete set
of benchmarks.  The one built-in bias is towards industrial rather than crafted
or random benchmarks.  This reflects a desire by the organizers and agreed upon
by the SMT community to emphasize problems that come from real applications.

Pseudo-random numbers will be generated using the standard C library
function \texttt{random()}, seeded (using \texttt{srandom()}) with the
sum, modulo $2^{30}$, of the numbers provided in the system
descriptions (see Section~\ref{sec:entrants} above) by all SMT-COMP
entrants other than the organizers.  Additionally, the integer part of
the opening value of the New York Stock Exchange Composite Index on
a publicized day
will be added to the other seeding values.  This helps provide transparency,
by guaranteeing that the organizers cannot manipulate the seed in
favor of or against any particular submitted solver.  Benchmarks will also be slightly
scrambled before the competition, using a simple benchmark scrambler
seeded with the same seed as the benchmark selector.  Both the
scrambler and benchmark selector will be publicly available before the
competition.  Naturally, solvers must not rely on previously
determined identifying syntactic characteristics of competition
benchmarks in testing satisfiability (violation of this is considered
cheating).

\subsection{Application track}

\header{Benchmark sources.} %
Benchmarks for the application track will be collected by the SMT-COMP organizers.
Any benchmark available to the organizers by April 15th (see the timeline in
Section~\ref{sec:timeline}) will be considered eligible.

\header{Benchmark availability.} 
A first release of the application track
benchmarks will be made available March 15, 2012.
More benchmarks can be collected, until April 15.
No additional
benchmarks will be added after this date, but benchmarks can be
modified or removed to fix possible bugs or other issues. 
The final release that will be used for the competition will be posted on 
June 1.

\header{Benchmark demographics and selection.} 
All the available benchmarks will be used for the competition. 
As was the case in 2011, no difficulty will be assigned to the benchmarks for the application track.
Benchmarks will be slightly scrambled before the competition, using the same scrambler and 
random seed as the main track. 
The selection algorithm for the application track will be chosen after the
 April 15th deadline, when the exact demographic of the benchmarks (collected with a usual
``call for benchmarks'') will be available.

\subsection{Benchmarks for the unsat-core and proof-generation tracks}

These benchmarks will be selected from the pool of available and newly submitted benchmarks,
adjusted to add formula naming for the unsat-core track. The organizers are still discussing
which logic divisions to use.
%%TODO: Divisions for unsat core and proof tracks

\section{Judging and Scoring}
\label{sec:judging}

\header{Main and application tracks}
The score for each benchmark is a triple $\langle e,n,m\rangle$, with
$e$ a non-negative number of erroneous results,
$n\in[0,N]$ an integral number of points scored for the benchmark,
where $N$ is the number of \akey{check-sat} commands
in the benchmark, and $m\in[0,T]$ is the (real-valued) time in seconds, where $T$ is
the timeout.  Recall that main track benchmarks will have just one \akey{check-sat} command;
application track benchmarks may have multiple \akey{check-sat} commands.
The score for the benchmark is initialized with
$\langle0,0,0\rangle$ and then computed as follows.
\begin{itemize}
\item A correctly-reported \texttt{sat} or \texttt{unsat} answer after
  $s$~seconds (counting from the beginning of this particular
  \akey{check-sat}) contributes $\langle0,1,s\rangle$ to the running
  score.
\item An answer of \texttt{unknown}, an unexpected answer, a crash, or a memory-out during
  execution of the query, or a benchmark timeout, aborts the execution
  of the benchmark and assigns the current value of the running score
  to the benchmark.  (Recall that there is one timeout for the entire
  benchmark; there are no individual timeouts for queries.)
\item An incorrect answer to a \akey{check-sat} command has the effect of terminating the
  evaluation of that individual benchmark, and the returned score for the benchmark
  will be $\langle1,0,0\rangle$.
\end{itemize}


For example, if a benchmark has 5~\akey{check-sat} commands, and a
timeout of 100~seconds, and a solver solves the first four in
10~seconds each, then times out on the fifth, then the solver's score
is $\langle0,4,40\rangle$.  If another solver solves each of the first
four in 10~seconds each, and the fifth in another 40~seconds, its
score is $\langle0,5,80\rangle$.  If a third solver solves the first
four queries in 2~seconds each, but incorrectly answers the fifth, its
score is $\langle1,0,0\rangle$.

As queries are only presented in order, this scoring system may mean
that relatively ``easier'' queries are hidden behind more difficult
ones located at the middle of the query sequence.

Benchmarks' scores are summed componentwise to form a solver's total
score for the competition.
Total scores are compared lexicographically---a score $\langle e,n,m\rangle$ is better than 
$\langle e',n',m'\rangle$ iff $e < e'$ or ($e = e'$ and $n > n'$) or ($e = e'$ and $n = n'$ and $m < m'$).
That is, fewer errors takes precedence over more correct solutions, which takes precedence over less time taken.

\header{Unsat-core track}

The unsat-core track will be scored as follows. The score for each benchmark is a triple $\langle e,n,m\rangle$, with
$e$ a non-negative number of erroneous results,
$n$ the reduction in number of formula in the unsat core for the benchmark,
and $m\in[0,T]$ is the (real-valued) time is seconds, where $T$ is
the timeout. 
The score for the benchmark is initialized with
$\langle0,0,0\rangle$ and then computed as follows.
\begin{itemize}
\item A correctly-reported unsat-core answer after
  $s$~seconds (counting from the beginning of this particular
  \akey{check-sat}) contributes $\langle0,n,s\rangle$ to the running
  score, where $n$ is the reduction in number of formula in the unsat core (number of named formula in the benchmark minus
  the number reported as the unsat core).
\item An answer of \texttt{unknown}, an unexpected answer, a crash, or a memory-out during
  execution of the query, or a benchmark timeout, aborts the execution
  of the benchmark and assigns a score of $\langle0,0,T\rangle$ for the benchmark, where
  $T$ is the timeout value.
  
\item The score for an incorrect answer to a \akey{check-sat} command 
  will be $\langle1,0,T\rangle$, where
  $T$ is the timeout value.
\end{itemize}

Scores are summed and ordered as for the main track.

\header{Number of competitors}
Winners in each competition Problem Division for which there are at least three
entrants from distinct research groups competing will be taken to be
those with the highest score and no erroneous results. In addition to recognizing the overall
winner in each division, the top solver that provides its source code will also be
recognized in each division.  
For Problem Divisions with fewer than three
entrants, the results will be reported but no winner officially
declared.


\section{Timeline}
\label{sec:timeline}

\nobreak
\vbox{% no page break here, please!
\begin{description}
\item[March 15] First version of the benchmark library for the application track posted for comment.
  First version of the benchmark scrambler, benchmark selector and trace exector made available.
\item[April 15] Final version of the benchmark scrambler, benchmark selector and trace exector made available.
\item[April 15] No new benchmarks can be added after this date, but
  problems with existing benchmarks may be fixed.
\item[June 1] Benchmark libraries are frozen.
\item[June 15 (7pm EDT)] Solvers due via SMTExec (for
  all tracks), including system
  descriptions and magic numbers for benchmark scrambling.
\item[June 18 (7pm EDT)] Final versions due, fixing any last-minute
 bugs (this marks the end of the grace period for
 submissions).
\item[June 20] Opening value of NYSE Composite Index used to complete random seed.
\item[June 25--29] Anticipated dates for SMT-COMP 2012.
\end{description}}

\section{Mailing List}
\label{sec:ps}

Interested parties should subscribe to the SMT-COMP mailing list, a
link to which is found at \url{www.smtcomp.org}.  Important
late-breaking news and any necessary clarifications and edits to these
rules will be announced there, and it is the primary way that such
announcements will be communicated.

\section{Disclaimer}
\begin{itemize}
\item David Cok is the chief organizer of SMT-COMP 2012. He is 
responsible for all policy and procedure decisions, such as these
rules. He is not associated 
with any group creating or submitting solvers. He has used solvers
in industrial settings and is keenly interested to know which are the best.

\item Alberto Griggio and Roberto Bruttomesso are co-organizers. They 
were also co-organizers in 2011. They are responsible for bringing 
recommendations from the previous year's experience. They will also 
be validating the competition setup, checking benchmarks, and
operating the competition. They have been associated with solver groups
that have submitted solvers to the competition, and may again in 2012 (in 2011, 
MathSat and OpenSMT, respectively).
\end{itemize}
\bibliographystyle{plain}
\bibliography{biblio}

\appendix
\section{Sample benchmark scripts for the main track}

\header{QF\_UF}

{\footnotesize
\begin{verbatim}
(set-logic QF_UF)
(set-info :status sat)
(declare-sort U 0)
(declare-fun f (U) U)
(declare-fun g (U) U)
(declare-fun A () Bool)
(declare-fun x () U)
(declare-fun y () U)
(assert
(let ((fx (f x))
      (cls1 (or A (= x y))))
  (and cls1 (distinct fx (g y)))))
(check-sat)
(exit)
\end{verbatim}}


\header{QF\_LRA}

{\footnotesize
\begin{verbatim}
(set-logic QF_LRA)
(declare-fun x () Real)
(declare-fun y () Real)
(declare-fun A () Bool)
(assert
  (let ((i1 (ite A (<= (+ (* 2.0 x) (* (/ 1 3) y)) (- 4))
                   (= (* y 1.5) (- 2 x)))))
    (and
      i1
      (or (> x y) (= A (< (* 3 x) (+ (- 1) (* (/ 1 5) (+ x y)))))))))
(check-sat)
(exit)
\end{verbatim}}


\header{QF\_LIA}

{\footnotesize
\begin{verbatim}
(set-logic QF_LIA)
(declare-fun x () Int)
(declare-fun y () Int)
(declare-fun A () Bool)
(assert
  (let ((i1 (ite A (<= (+ (* 2 x) (* (- 1) y)) (- 4))
                   (= (* y 5) (- 2 x)))))
    (and
      i1
      (or (> x y) (= A (< (* 3 x) (+ (- 1) (* 1 (+ x y)))))))))
(check-sat)
(exit)
\end{verbatim}}


\header{QF\_BV}

{\footnotesize
\begin{verbatim}
(set-logic QF_BV)
(declare-fun x () (_ BitVec 32))
(declare-fun y () (_ BitVec 16))
(declare-fun z () (_ BitVec 20))
(assert
  (let ((c1 (= x ((_ sign_extend 12) z))))
   (let ((c2 (= y ((_ extract 18 3) x))))
    (let ((c3 
            (bvslt (concat z (_ bv5 12)) 
              (bvand (bvor (bvxor (bvnot x) ((_ zero_extend 28) #b1111)) 
                                    (concat #xAF02 y))
                    (concat (bvmul ((_ extract 31 16) x) y) 
                            (bvashr (_ bv42 16) #x0001))))))
   (and c1 (xor c2 c3))))))
(check-sat)
(exit)
\end{verbatim}}


\header{QF\_AUFLIA}

{\footnotesize
\begin{verbatim}
(set-logic QF_AUFLIA)
(declare-fun A () (Array Int Int))
(declare-fun x () Int)
(declare-fun y () Int)
(declare-fun P () Bool)
(declare-sort U 0)
(declare-fun f (U) (Array Int Int))
(declare-fun c () U)
(assert
  (let ((fc (f c)))
    (and
      (=> (= A (store fc x 5)) (> (+ (select fc y) (* 4 x)) 0))
      (= P (< (select A (+ 3 y)) (* (- 2) x))))))
(check-sat)
(exit)
\end{verbatim}}


\header{QF\_ABV}

\footnotesize
\begin{verbatim}
(set-logic QF_ABV)
(declare-fun x () (_ BitVec 32))
(declare-fun y () (_ BitVec 16))
(declare-fun z () (_ BitVec 20))
(declare-fun A () (Array (_ BitVec 16) (_ BitVec 32)))
(assert
  (let ((c1 (= ((_ sign_extend 12) z) (select A y)))
        (A2 (store A ((_ extract 15 0) x) x)))
   (let ((c2 (= A A2)))
    (let ((c3 
            (bvslt (concat z (_ bv5 12)) 
              (bvand (bvor (bvxor (bvnot x) 
                                  (select A2 ((_ zero_extend 12) #b1111)))
                                    (concat #xAF02 y))
                    (concat ((_ extract 15 0) 
                                (bvmul x (select (store A y x) #x35FB))) 
                            (bvashr (_ bv42 16) #x0001))))))
   (and c1 (xor c2 c3))))))
(check-sat)
(exit)
\end{verbatim}

\section{Sample benchmark scripts for the application track}

\footnotesize
\begin{verbatim}
(set-option :print-success true)
(set-logic QF_LRA)
(declare-fun c0 () Bool)
(declare-fun E0 () Bool)
(declare-fun f0 () Bool)
(declare-fun f1 () Bool)
(push 1)
(assert 
  (let ((.def_10 (not f0)))
   (let ((.def_9 (not c0)))
    (let ((.def_11 (or .def_9 .def_10)))
     (let ((.def_7 (not f1)))
      (let ((.def_8 (or c0 .def_7)))
       (let ((.def_12 (and .def_8 .def_11)))
 .def_12
)))))))
(check-sat)
(pop 1)
(declare-fun f2 () Bool)
(declare-fun f3 () Bool)
(declare-fun f4 () Bool)
(declare-fun c1 () Bool)
(declare-fun E1 () Bool)
(assert 
 (let ((.def_23 (not f2)))
  (let ((.def_20 (= c0 c1)))
   (let ((.def_22 (or E0 .def_20)))
    (let ((.def_24 (or .def_22 .def_23)))
     (let ((.def_18 (not f4)))
      (let ((.def_19 (or c1 .def_18)))
       (let ((.def_25 (and .def_19 .def_24)))
 .def_25
))))))))
(push 1)
(check-sat)
(assert (and f1 (not f1)))
(check-sat)
(pop 1)
(exit)
\end{verbatim}

\end{document}
