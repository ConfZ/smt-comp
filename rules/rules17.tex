\documentclass[12pt]{article}

\usepackage{color}
\usepackage{times}
\usepackage{fullpage}
\usepackage{hyperref}

\usepackage{draftwatermark}

\hyphenation{data-types}

\newcommand{\akey}[1]{\textbf{#1}}

\let\OLDthebibliography\thebibliography
\renewcommand\thebibliography[1]{
  \OLDthebibliography{#1}
  \setlength{\parskip}{2pt}
  \setlength{\itemsep}{1.5pt plus 0.3ex}
}

%%%%%%%%%%%%%%%%%%%%%%%%%%%%%%%%%%%%%%%%%%%%%%%%%%%%%%%%%%%%%%%%%%%%%%%%%%%%%%%%

\begin{document}

\date{\small This version revised \the\year-\the\month-\the\day}

\title{12th International Satisfiability Modulo Theories Competition
  (SMT-COMP 2017): Rules and Procedures}

% [morgan] do our own layout of authors; the four-author layout spacing
% was screwed up...
\def\doauthor#1{{%
  \hsize.5\hsize \advance\hsize by-1cm %
  \def\\{\hss\egroup\hbox to\hsize\bgroup\strut\hss}%
  \vbox{\hbox to\hsize\bgroup\strut\hss#1\hss\egroup}}}%

\def\header#1{\medskip\noindent\textbf{#1}}

\author{%
Matthias Heizmann \\
University of Freiburg \\
Germany \\
{\small\href{mailto:heizmann@informatik.uni-freiburg.de}{\texttt{heizmann@informatik.uni-freiburg.de}}} \\
\and
Giles Reger \\
University of Manchester \\
UK \\
{\small\href{mailto:giles.reger@manchester.ac.uk}{\texttt{giles.reger@manchester.ac.uk}}} \\
\and
Tjark Weber \\
Uppsala University \\
Sweden \\
{\small\href{mailto:tjark.weber@it.uu.se}{\texttt{tjark.weber@it.uu.se}}} \\
\\
}

\maketitle

\noindent Comments on this document should be emailed to the SMT-COMP
mailing list (see below) or, if necessary, directly to the organizers.

%%%%%%%%%%%%%%%%%%%%%%%%%%%%%%%%%%%%%%%%%%%%%%%%%%%%%%%%%%%%%%%%%%%%%%%%%%%%%%%%

\section{Communication}

Interested parties should subscribe to the SMT-COMP mailing list.
Important late-breaking news and any necessary clarifications and
edits to these rules will be announced there, and it is the primary
way that such announcements will be communicated.

\begin{itemize}
\item SMT-COMP mailing list:
  \href{mailto:smt-comp@cs.nyu.edu}{\texttt{smt-comp@cs.nyu.edu}}
\item Sign-up site for the mailing list:
  \url{cs.nyu.edu/mailman/listinfo/smt-comp}
\end{itemize}

\noindent Additional material will be made available at the
competition web site, \url{www.smtcomp.org} or
\url{smtcomp.sourceforge.net/2017}.

%%%%%%%%%%%%%%%%%%%%%%%%%%%%%%%%%%%%%%%%%%%%%%%%%%%%%%%%%%%%%%%%%%%%%%%%%%%%%%%%

\newpage

\section{Important Dates}
\label{sec:important}

\begin{description}
\item[May~1] Deadline for new benchmark contributions.
\item[June~1] Final versions of competition tools (e.g., benchmark
  scrambler) are made available.  Benchmark libraries are frozen.
\item[June~4] Deadline for first versions of solvers (for all tracks),
  including information about which tracks and divisions are being
  entered, and magic numbers for benchmark scrambling.
\item[June~18] Deadline for final versions of solvers, including
  system descriptions.
\item[June~19] Opening value of NYSE Composite Index used to complete
  random seed for benchmark scrambling.
\item[July~22/23] SMT Workshop; end of competition, presentation of
  results.
\end{description}

%%%%%%%%%%%%%%%%%%%%%%%%%%%%%%%%%%%%%%%%%%%%%%%%%%%%%%%%%%%%%%%%%%%%%%%%%%%%%%%%

\section{Introduction}

The annual Satisfiability Modulo Theories Competition~(SMT-COMP) is
held to spur advances in SMT solver implementations on benchmark
formulas of practical interest.  Public competitions are a well-known
means of stimulating advancement in software tools.  For example, in
automated reasoning, the CASC and SAT competitions for first-order and
propositional reasoning tools, respectively, have spurred significant
innovation in their fields~\cite{leberre+03,PSS02}.  More information
on the history and motivation for SMT-COMP can be found at the
competition web site, \url{www.smtcomp.org}, and in reports on
previous
competitions~(\cite{SMTCOMP-JAR,SMTCOMP-FMSD,BDOS08,SMTCOMP-2008,CDW14,SMTCOMP-2012,CSW15}).

SMT-COMP~2017 is part of the SMT Workshop~2017
(\url{http://smt-workshop.cs.uiowa.edu/2017/}), which is affiliated
with CAV~2017 (\url{http://cavconference.org/2017/}).  The SMT
Workshop will include a block of time to present the results of the
competition.

Accordingly, researchers are highly encouraged to submit both new
benchmarks and new or improved solvers to raise the level of
competition and advance the state-of-the-art in automated SMT problem
solving.

SMT-COMP 2017 will have three tracks: the conventional main track, an
application (i.e., incremental) track, and an unsat-core track.

Within each track there are multiple divisions, where each division
uses benchmarks from a specific SMT-LIB logic (or group of logics).
We will recognize winners as measured by number of benchmarks solved;
we will also recognize solvers based on additional criteria.

The rest of this document, revised from the previous
version,\footnote{Earlier versions of this document include
  contributions from Clark Barrett, Roberto Bruttomesso, David Cok,
  Sylvain Conchon, David D{\'e}harbe, Morgan Deters, Alberto Griggio,
  Matthias Heizmann, Albert Oliveras, Aaron Stump, and Tjark Weber.}
describes the rules and competition procedures for SMT-COMP~2017.  The
principal changes from the previous competition rules are the
following:
\begin{itemize}
\item Submission of accompanying information for solvers is via a web
  form (rather than by email to the organizers).  \emph{Rationale:}
  Email submissions were often incomplete, leading to ambiguity and
  further inquiries.  The form provides a more structured way of
  submitting information.
\item Benchmarks may use (a subset of) version~2.5 of the SMT-LIB
  language.  \emph{Rationale:} This is the latest version of the
  SMT-LIB language.  Its use had been announced for SMT-COMP 2016
  already, but was delayed because the SMT-LIB benchmark collection
  had not been updated.  Solvers should ideally support both
  versions~2.0 and~2.5; the latter is largely backwards-compatible.
\item Benchmarks that use partial bit-vector functions, such as
  \texttt{bvudiv}, or underspecified floating-point functions, such as
  \texttt{fp.min}, may become eligible for the competition again,
  provided (i)~the semantics of these functions agrees with existing
  solver implementations, and (ii)~the status of affected benchmarks
  in SMT-LIB is not incorrect.  \emph{Rationale:} The semantics of
  these functions had been under discussion, and it appears that a
  conclusion has finally been reached.
\item Benchmarks with unknown status will be eligible for the
  competition's main track.  In 2016, solver performance on benchmarks
  with unknown status was evaluated but reported separately; in 2017,
  it will directly affect the competition results.  \emph{Rationale:}
  Using only benchmarks with known status (i.e., benchmarks that have
  been solved before) to determine the competition results unjustly
  favors imitation of existing solvers over true innovation.
\item In the competition-wide scoring, the (per-division) penalty for
  erroneous results is changed from the number of errors to a fixed
  constant (multiplied by the weight of the respective division).
  \emph{Rationale:} The weight already takes the number of benchmarks
  in the division into account; multiplying it by the number of errors
  overemphasizes large divisions.
\item The unsat-core track, which was re-introduced in 2016, is no
  longer experimental.  Its divisions will be considered competitive
  if the necessary requirements (see Section~\ref{sec:scoring}) are
  met.  \emph{Rationale:} Rules and tool support for the unsat-core
  track proved sufficiently stable in 2016.
\item The competition may feature experimental divisions for
  benchmarks that use algebraic datatypes, provided suitable
  benchmarks are available.  \emph{Rationale:} Algebraic datatypes are
  specified in the draft SMT-LIB~2.6 standard, which is expected to
  become official in the near future, and are already supported by
  some solvers.
\end{itemize}

%%%%%%%%%%%%%%%%%%%%%%%%%%%%%%%%%%%%%%%%%%%%%%%%%%%%%%%%%%%%%%%%%%%%%%%%%%%%%%%%

\section{Entrants}
\label{sec:entrants}

\header{Solver submission.}
%
An entrant to SMT-COMP is an SMT solver submitted by its authors using
the StarExec (\url{http://www.starexec.org}) service.  The execution
service enables members of the SMT research community to run solvers
on jobs consisting of benchmarks from the SMT-LIB benchmark library.
Jobs are run on a shared computer cluster.  The execution service is
provided free of charge, but it does require registration to create a
login account.  Registered users may then upload their own solvers to
run, or may run public solvers already uploaded to the service.
Information about how to configure and upload a solver is contained in
the StarExec user guide,
\url{https://wiki.uiowa.edu/display/stardev/User+Guide}.

For participation in SMT-COMP, a solver must be uploaded to StarExec
and made publicly available.  StarExec supports solver configurations;
for clarity, \emph{each submitted solver must have one configuration
  only}.  Moreover, the organizers must be informed of the solver's
presence \emph{and the tracks and divisions in which it is
  participating} via the web form at
\begin{center}
  \url{https://goo.gl/forms/HL6rbOywqW4sFo1D2}
\end{center}
A submission must also include a 32-bit unsigned integer.  These
integer numbers, collected from all submissions, are used to seed the
benchmark scrambler.

\header{System description.}
%
As part of a submission, SMT-COMP entrants are encouraged to provide a
short (1--2 pages) description of the system.  This should include a
list of all authors of the system, their present institutional
affiliations, and any appropriate acknowledgements.  The programming
language(s) and basic SMT solving approach employed should be
described (e.g., lazy integration of a Nelson-Oppen combination with
SAT, translation to SAT, etc.).  System descriptions are encouraged to
include a URL for a web site for the submitted tool.  System
descriptions may be submitted after the solver deadline, but to be
useful should be sent to the organizers before the competition ends.
We intend to make system descriptions publicly available.

\header{Multiple versions.}
%
The organizers' intent is to promote as wide a comparison among
solvers and solver options as possible.  However, if the number of
solver submissions is too large for the computational resources
available to the competition, the organizers reserve the right not to
accept multiple versions of solvers from the same solver team.

\header{Other solvers.}
%
The organizers reserve the right to include other solvers of interest
(such as entrants in previous SMT competitions) in the competition,
e.g., for comparison purposes.

\header{Wrapper tools.}
%
A \emph{wrapper tool} is defined as any solver which calls one or more
SMT solvers not written by the author of the wrapper tool.  The other
solvers are called the \emph{wrapped} solvers.  A wrapper tool must
explicitly acknowledge any solvers that it wraps.  Its system
description should make clear the technical innovations by which the
wrapper tool expects to improve on the wrapped solvers.

\header{Attendance.}
%
Submitters of an SMT-COMP entrant need not be physically present at
the competition or the SMT Workshop to participate or win.

%%%%%%%%%%%%%%%%%%%%%%%%%%%%%%%%%%%%%%%%%%%%%%%%%%%%%%%%%%%%%%%%%%%%%%%%%%%%%%%%

\subsection*{Deadlines}

SMT-COMP entrants must be submitted via StarExec (solvers) \emph{and}
the above web form (accompanying information) until the end of {\bf
  June~4, 2017} anywhere on earth.  After this date no new entrants
will be accepted.  However, updates to existing entrants on StarExec
will be accepted until the end of {\bf June~18, 2017} anywhere on
earth.

We strongly encourage participants to use this grace period
\emph{only} for the purpose of fixing any bugs that may be discovered,
and not for adding new features, as there may be no opportunity to do
extensive testing using StarExec after the initial deadline.

The solver versions that are present on StarExec at the conclusion of
the grace period will be the ones used for the competition.  Versions
submitted after this time will not be used.  The organizers reserve
the right to start the competition itself at any time after the open
of the New York Stock Exchange on the day after the final solver
deadline.

These deadlines and procedures apply equally to all tracks of the
competition.

%%%%%%%%%%%%%%%%%%%%%%%%%%%%%%%%%%%%%%%%%%%%%%%%%%%%%%%%%%%%%%%%%%%%%%%%%%%%%%%%

\section{Execution of Solvers}

Solvers will be publicly evaluated in all tracks and divisions into
which they have been entered.  All results of the competition will be
made public.

\subsection{Logistics}
\label{sec:logistics}

\header{Dates of competition.}
%
The bulk of the computation will take place during the weeks leading
up to SMT 2017.  Intermediate results will be regularly posted to the
SMT-COMP website as the competition runs.

The organizers reserve the right to prioritize certain competition
tracks or divisions to ensure their timely completion, and in
exceptional circumstances to complete divisions after the SMT
Workshop.

\header{Input and output.}
%
In the main and unsat-core track, a participating solver must read a
single benchmark script, whose filename is presented as the solver's
first command-line argument.  In the application track, a trace
executor will send commands from a benchmark script to the solver's
standard input channel.

The benchmark script is in the concrete syntax of the SMT-LIB format,
version~2.5, though with a restricted set of commands.  A benchmark
script is a text file containing a sequence of SMT-LIB commands that
satisfies the following requirements:
%
\begin{enumerate}
\item
  \begin{enumerate}
  \item In the main and unsat-core track, there may be a single
    \akey{set-option :print-success ...} command.  Note that
    \texttt{success} outputs are ignored by the post-processor used by
    the competition.\footnote{SMT-LIB~2.5 requires solvers to produce
      a \texttt{success} answer after each \akey{set-logic},
      \akey{declare-sort}, \akey{declare-fun} and \akey{assert}
      command (among others), unless the option \akey{:print-success}
      is set to false.  Ignoring the \texttt{success} outputs allows
      for submitting fully SMT-LIB~2.5 compliant solvers without the
      need for a wrapper script, while still allowing entrants of
      previous competitions to run without changes.}
  \item In the application track, the \akey{:print-success} option
    must not be disabled.  The trace executor will send an initial
    \akey{set-option :print-success true} command to the solver.
  \end{enumerate}
\item In the unsat-core track, there is a single \akey{set-option
  :produce-unsat-cores true} command.
\item The (single) \akey{set-logic} command setting the benchmark's
  logic is the first command after any \akey{set-option} commands.
\item The script may contain any number of \akey{set-info} commands.
\item The script may contain any number of \akey{declare-sort} and
  \akey{define-sort} commands.  All sorts declared or defined with
  these commands must have zero arity.
\item The script may contain any number of \akey{declare-fun} and
  \akey{define-fun} commands.
\item The script may contain any number of \akey{assert} commands.
  All formulas in the script belong to the benchmark's logic, with any
  free symbols declared in the script.
\item
  \begin{enumerate}
    \item In the main and application track, named formulas are not
      used.
    \item In the unsat core track, top-level assertions may be named.
  \end{enumerate}
\item
  \begin{enumerate}
  \item In the main and unsat-core track, there is exactly one
    \akey{check-sat} command.
  \item In the application track, there are one or more
    \akey{check-sat} commands.  There may also be zero or more
    \akey{push 1} commands, and zero or more \akey{pop 1} commands,
    consistent with the use of those commands in the SMT-LIB standard.
  \end{enumerate}
\item In the unsat-core track, the \akey{check-sat} command (which is
  always issued in an unsatisfiable context) is followed by a single
  \akey{get-unsat-core} command.
\item The script may optionally contain an \akey{exit} command as its
  last command.  In the application track, this command must not be
  omitted.
\item No other commands besides the ones just mentioned may be used.
\end{enumerate}
%
The SMT-LIB format specification is available from the ``Standard''
section of the SMT-LIB website~\cite{SMT-LIB}.  Solvers will be given
formulas only from the divisions into which they have been entered.

\header{Time and memory limits.}
%
Each SMT-COMP solver will be executed on a dedicated processor of a
competition machine, for each given benchmark, up to a fixed
wall-clock time limit~$T$.  Each processor has 4 cores.  Detailed
machine specifications are available on the competition web site.

The time limit~$T$ is yet to be determined, but it is anticipated to
be at most 40~minutes of wall-clock time per solver/benchmark
pair.\footnote{The time limit may be adjusted once we know the number
  of competition entrants and eligible benchmarks.  Because of the
  inclusion of benchmarks with unknown status, the time limit may be
  lower than in previous years.}  Solvers that take more than this
time limit will be killed.  Solvers are allowed to spawn other
processes; these will be killed at approximately the same time as the
first started process.

The StarExec service also limits the memory consumption of the solver
processes.  We expect the memory limit per solver/benchmark pair to be
on the order of 60\,GB.  The values of both the time limit and the
memory limit are available to a solver process through environment
variables.  See the StarExec user guide for more information.

\subsection{Main track}
\label{sec:exec:main}

The main track competition will consist of selected benchmarks in each
of the logic divisions.  Each benchmark script will be presented to
the solver as its first command-line argument.  The solver is then
expected to attempt to report on its standard output channel whether
the formula is satisfiable (\texttt{sat}, in lowercase) or
unsatisfiable (\texttt{unsat}).  A solver may also report
\texttt{unknown} to indicate that it cannot determine satisfiability
of the formula.

The main track competition uses a StarExec post-processor (named
``SMT-COMP 2017") to accumulate the results.

\header{Aborts and unparsable output.}
%
Any \texttt{success} outputs will be ignored.  Solvers that exit
before the time limit without reporting a result (e.g., due to
exhausting memory or crashing) \emph{and} do not produce output that
includes \texttt{sat}, \texttt{unsat} or \texttt{unknown} will be
considered to have aborted.

\header{Persistent state.}
%
Solvers may create and write to files and directories during the
course of an execution, but they must not read such files back during
later executions.  Each solver is executed with a temporary directory
as its current working directory.  Any generated files should be
produced there (and not, say, in the system's \texttt{/tmp}
directory).  The StarExec system sets a limit on the amount of disk
storage permitted---typically 20\,GB.  See the StarExec user guide for
more information.  The temporary directory is deleted after the job is
complete.  Solvers must not attempt to communicate with other
machines, e.g., over the network.

\subsection{Application track}

The application track evaluates SMT solvers when interacting with an
external verification framework, e.g., a model checker. This
interaction, ideally, happens by means of an online communication
between the framework and the solver: the framework repeatedly sends
queries to the SMT solver, which in turn answers either \texttt{sat}
or \texttt{unsat}.  In this interaction an SMT solver is required to
accept queries incrementally via its \emph{standard input channel}.

In order to facilitate the evaluation of solvers in this track, we
will set up a ``simulation'' of the aforementioned interaction.  Each
benchmark in the application track represents a realistic
communication trace, containing multiple \akey{check-sat} commands
(possibly with corresponding \akey{push 1} and \akey{pop 1} commands),
which is parsed by a \emph{trace executor}. The trace executor serves
the following purposes:
\begin{itemize}
\item it simulates the online interaction by sending single queries to
  the SMT solver (through stdin);
\item it prevents ``look-ahead'' behaviors of SMT solvers;
\item it records time and answers for each command;
\item it guarantees a fair execution for all solvers by abstracting
  from any possible crash, misbehavior, etc.\ that might happen in the
  verification framework.
\end{itemize}
%
The trace executor terminates processing the benchmark script upon
receiving an incorrect response from the solver.

The disk space and memory limits for the application track are the
same as for the main track (see Section~\ref{sec:exec:main}).

\header{Input and output.}
%
Participating solvers will be connected to a trace executor, which
will incrementally send commands to the standard input channel of the
solver and read responses from the standard output channel of the
solver.  The commands will be taken from an SMT-LIB benchmark script
that satisfies the requirements for application track scripts given in
Section~\ref{sec:logistics}.

Solvers must respond to each command sent by the trace executor with
the answers defined in the SMT-LIB format specification, that is, with
an answer of \texttt{sat}, \texttt{unsat}, or \texttt{unknown} for
\akey{check-sat} commands, and with a \texttt{success} answer for
other commands.

\subsection{Unsat-core track}
\label{sec:exec:unsat-core}

The unsat-core track will evaluate the capability of solvers to
generate unsatisfiable cores (for problems that are known to be
unsatisfiable).  Solvers will be measured by the smallness of the
unsatisfiable core they return.

The SMT-LIB language accommodates this functionality by providing two
features: the ability to name top-level (asserted) formulas, and the
ability to request an unsatisfiable core after a \akey{check-sat}
command returns \texttt{unsat}.  The unsatisfiable core that is
returned must consist of a list of names of formulas, in the format
prescribed by the SMT-LIB standard.

The result of a solver is considered erroneous if the response to the
\akey{check-sat} command is \texttt{sat}, or if the returned
unsatisfiable core is not well-formed (e.g., contains names of
formulas that have not been asserted before), or if the returned
unsatisfiable core is not, in fact, unsatisfiable.

In order to perform this unsatisfiability check, the organizers will
use a selection of SMT solvers that have been sound (i.e., not
produced any erroneous result) in the corresponding division of the
main track.  The unsatisfiability of a produced unsatisfiable core is
refuted if the number of checking solvers whose result is \texttt{sat}
exceeds the number of checking solvers whose result is \texttt{unsat},
or if the unsat-core producing solver is used as a checking solver and
its result is \texttt{sat}.  The time limit for checking unsatisfiable
cores is yet to be determined, but is anticipated to be around 5
minutes of wall-clock time per solver.

Solvers must respond to each command in the benchmark script with the
answers defined in the SMT-LIB format specification.  In particular,
solvers that respond \texttt{unknown} to the \akey{check-sat} command
must respond with an error to the following \akey{get-unsat-core}
command.

%%%%%%%%%%%%%%%%%%%%%%%%%%%%%%%%%%%%%%%%%%%%%%%%%%%%%%%%%%%%%%%%%%%%%%%%%%%%%%%%

\section{Benchmarks and Problem Divisions}

\header{Benchmark sources.}
%
Benchmarks for each division will be drawn from the SMT-LIB benchmark
library.  The main track will use a subset of all
\emph{non-incremental} benchmarks; the application track will use a
subset of all \emph{incremental} benchmarks.  The unsat-core track
will use unsatisfiable main track benchmarks, modified to use named
top-level assertions.

\header{New benchmarks.}
%
The deadline for submission of new benchmarks is {\bf May~1, 2017}.
The organizers, in collaboration with the SMT-LIB maintainers, will be
checking and curating these until {\bf June~1, 2017}.  The SMT-LIB
maintainers intend to make a new release of the benchmark library
publicly available on or close to this date.

\header{Benchmark demographics.}
%
In SMT-LIB, benchmarks are organized according to \emph{families}.  A
benchmark family contains problems that are similar in some
significant way.  Typically they come from the same source or
application, or are all output by the same tool.  Each top-level
subdirectory within a division represents a distinct family.

\header{Benchmark selection.}
%
The competition will use a large subset of SMT-LIB benchmarks.  The
benchmark pool is culled as follows:
\begin{enumerate}
\item \emph{Remove inappropriate benchmarks.} The competition
  organizers may remove benchmarks that are deemed
  inappropriate\footnote{In previous years, this included all
    benchmarks that use partial or underspecified functions, e.g.,
    \texttt{bvudiv}, \texttt{fp.min}.  Whether these benchmarks are
    included in 2017 is yet to be decided.} or uninteresting for
  competition, or cut the size of certain benchmark families to avoid
  their over-representation.  SMT-COMP attempts to give preference to
  benchmarks that are ``real-world,'' in the sense of coming from or
  having some intended application outside SMT.
\item \emph{Remove incremental benchmarks whose first \akey{check-sat}
  command has unknown status.}  Incremental benchmarks may contain
  multiple \akey{check-sat} commands, each with its own status.  If an
  incremental benchmark contains a \akey{check-sat} command whose
  status is unknown, only the prefix of the benchmark up to the
  preceding \akey{check-sat} command is eligible for the application
  track of the competition; if the benchmark's first \akey{check-sat}
  command has unknown status, the entire benchmark is
  ineligible.\footnote{It might be desirable to also include
    \akey{check-sat} commands with unknown status in the application
    track (for the same reasons they are now included in the main
    track), but this would require substantial changes to the trace
    executor and scoring tools.}
\end{enumerate}
%
All remaining benchmarks are used for the competition.  There will be
no further selection of benchmarks, e.g., based on benchmark
difficulty or benchmark category.

The set of benchmarks selected for the competition will be published
when the competition begins.

\header{Heats.}
%
Since the organizers at this point are unsure how long the set of
benchmarks may take (which will depend also on the number of solvers
submitted), the competition may be run in \emph{heats}.  For each
track and division, the selected benchmarks may be randomly divided
into a number of (possibly unequal-sized) heats.  Heats will be run in
order.  If the organizers determine that there is adequate time, all
heats will be used for the competition.  Otherwise, incomplete heats
will be ignored.

\header{Benchmark scrambling.}
%
Benchmarks will be slightly scrambled before the competition, using a
simple benchmark scrambler.  The benchmark scrambler will be made
publicly available before the competition.

Naturally, solvers must not rely on previously determined identifying
syntactic characteristics of competition benchmarks in testing
satisfiability.  Violation of this rule is considered cheating.

\header{Pseudo-random numbers.}
%
Pseudo-random numbers used, e.g., for the creation of heats or the
scrambling of benchmarks, will be generated using the standard C
library function \texttt{random()}, seeded (using \texttt{srandom()})
with the sum, modulo $2^{30}$, of the integer numbers provided in the
system descriptions (see Section~\ref{sec:entrants}) by all SMT-COMP
entrants other than the organizers'.  Additionally, the integer part
of the opening value of the New York Stock Exchange Composite Index on
the first day the exchange is open on or after the date specified in
the timeline (Section~\ref{sec:important}) will be added to the other
seeding values.  This helps provide transparency, by guaranteeing that
the organizers cannot manipulate the seed in favor of or against any
particular submitted solver.

%%%%%%%%%%%%%%%%%%%%%%%%%%%%%%%%%%%%%%%%%%%%%%%%%%%%%%%%%%%%%%%%%%%%%%%%%%%%%%%%

\section{Scoring}
\label{sec:scoring}

\subsection{Competitive divisions}
  
Scores will be computed for all solvers and divisions.  However,
winners will be declared only for \emph{competitive} divisions.  A
division in a track is competitive if at least two substantially
different solvers (i.e., solvers from two different teams) were
submitted.  Although the organizers may enter other solvers for
comparison purposes, only solvers that are explicitly submitted by
their authors determine whether a division is competitive, and are
eligible to be designated as winners.

\subsection{Benchmark scoring}
\label{sec:benchmark-scoring}

A solver's \emph{raw score} for each benchmark is a quadruple $\langle
e, n, w, c\rangle$, with $e \in \{0, 1\}$ the number of erroneous
results (usually~$e = 0$), $0 \leq n \leq N$ the number of correct
results,\footnote{Here, $N$ is the number of \akey{check-sat} commands
  in the benchmark.  Recall that main track benchmarks have just one
  \akey{check-sat} command; application track benchmarks may have
  multiple \akey{check-sat} commands.} $w \in [0,T]$ the (real-valued)
wall-clock time in seconds, and $c \in [0, 4T]$ the (real-valued) CPU
time in seconds, measured across all cores and sub-processes, until
the solver process terminates.

\header{Main track.} More specifically, for the main track, we have
%
\begin{itemize}
\item $e=0$, $n=0$ if the solver aborts without a response, or the
  result of the \akey{check-sat} command is \texttt{unknown},
\item $e=0$, $n=1$ if the result of the \akey{check-sat} command is
  \texttt{sat} or \texttt{unsat}, and the result either agrees with
  the benchmark status or the benchmark status is unknown,\footnote{If
    the benchmark status is unknown, we thus treat the solver's answer
    as correct.  Disagreements between different solvers on benchmarks
    with unknown status are governed in
    Section~\ref{sec:division-scoring}.}
\item $e=1$, $n=0$ if the result of the \akey{check-sat} command is
  incorrect.
\end{itemize}
%
Note that a (correct or incorrect) response is taken into
consideration even when the solver process terminates abnormally, or
does not terminate within the time limit.  Solvers should take care
not to accidentally produce output that contains \texttt{sat} or
\texttt{unsat}.

\header{Application track.}
%
An application benchmark may contain multiple \akey{check-sat}
commands.  Solvers may partially solve the benchmark before timing
out.  The benchmark is run by the trace executor, measuring the total
time (summed over all individual commands) taken by the solver to
respond to commands.\footnote{Times measured by StarExec may include
  time spent in the trace executor.  We expect that this time will
  likely be insignificant compared to time spent in the solver, and
  nearly constant across solvers.}  Most time will likely be spent in
response to \akey{check-sat} commands, but \akey{assert}, \akey{push}
or \akey{pop} commands might also entail a reasonable amount of
processing.  For the application track, we have
\begin{itemize}
\item $e=1$, $n=0$ if the solver returns an incorrect result for any
  \akey{check-sat} command within the time limit,
\item otherwise, $e=0$, and $n$ is the number of correct results for
  \akey{check-sat} commands returned by the solver before the time
  limit is reached.
\end{itemize}

\header{Unsat-core track.}
%
For the unsat-core track, we instead have $0 \leq n \leq A$, where~$A$
is the number of named top-level assertions in the benchmark, and
\begin{itemize}
\item $e=0$, $n=0$ if the solver aborts without a response, or the
  result of the \akey{check-sat} command is \texttt{unknown},
\item $e=1$, $n=0$ if the result is erroneous according to
  Section~\ref{sec:exec:unsat-core},
\item otherwise, $e=0$, and $n$ is the reduction in the number of
  formulas, i.e., $n = A$ minus the number of formula names in the
  reported unsatisfiable core.
\end{itemize}

\subsection{Sequential performance (main track)}
\label{sec:sequential}

SMT-COMP has traditionally emphasized sequential performance (i.e.,
CPU time) over parallel performance (i.e., wall-clock time).  StarExec
measures both, and we intend to recognize both best sequential and
best parallel solvers in all competitive main track divisions.

The raw score, as defined in Section~\ref{sec:benchmark-scoring},
favors parallel solvers, which may utilize all available processor
cores.  To evaluate sequential performance, we derive a
\emph{sequential score} by imposing a (virtual) CPU time limit equal
to the wall-clock time limit~$T$.  A solver result is taken into
consideration for the sequential score only if the solver process
terminates within this CPU time limit.  More specifically, for a given
raw score $\langle e, n, w, c\rangle$, the corresponding sequential
score is defined as~$\langle e_S, n_S, c_S\rangle$, where
\begin{itemize}
\item $e_S = 0$ and $n_S = 0$ if $c > T$, $e_S = e$ and $n_S = n$
  otherwise,
\item $c_S = \min\ \{c, T\}$.\footnote{\emph{Rationale:} Under this
  score, a solver should not benefit from using multiple processor
  cores.  Conceptually, the sequential score should be (nearly)
  unchanged if the solver was run on a single-core processor, up to a
  time limit of~$T$.}
\end{itemize}

\subsection{Division scoring}
\label{sec:division-scoring}

\header{Main track: removal of disagreements.}
%
Before division scores are computed for the main track, benchmarks
with unknown status are removed from the competition results if two
(or more) solvers that are sound on benchmarks with known status
disagree on their result.  More specifically, a solver (including a
solver that was entered into the competition by the organizers for
comparison purposes) is \emph{sound on benchmarks with known status}
for a division if its raw score (Section~\ref{sec:benchmark-scoring})
is of the form $\langle 0, n, w, c\rangle$ for each benchmark in the
division, i.e., if it did not produce any erroneous results.  Two
solvers \emph{disagree} on a benchmark if one of them reported
\texttt{sat} and the other reported \texttt{unsat}.  Only the
remaining benchmarks are used in the following computation of division
scores.

\medskip

To compute a solver's score for a division, the solver's individual
benchmark scores for all benchmarks in the division are first
multiplied by a scalar weight that depends on the benchmark's family,
and then summed component-wise.

For a given competition benchmark~$b$, let~$F_b \geq 1$ be the total
number of benchmarks in $b$'s benchmark family that were used in the
competition track to which the division belongs (and not removed
because of disagreements).  We define the weight for benchmark~$b$ as
$\alpha_b = (1 + \log_e F_b) / F_b$.\footnote{See
  Section~\ref{sec:competition-wide} for a motivating discussion of
  log scaling.}  We define the \emph{normalized weight} for
benchmark~$b$ as $\alpha'_b = \alpha_b / (\sum_{b'} \alpha_{b'})$,
where the sum is over all benchmarks in the division.  Let~$N$ be the
total number of benchmarks in the division.

For main track and unsat-core track divisions, we will separately
compute the weighted sum of all raw scores
(Section~\ref{sec:benchmark-scoring}) $$\sum_b \alpha'_b \cdot \langle
e_b \cdot N, n_b \cdot N, w_b, c_b\rangle$$ where the sum is over all
benchmarks in the division to assess parallel performance, and the
weighted sum of all sequential scores (Section~\ref{sec:sequential})
to assess sequential performance.  For application track divisions,
division scores will be based on raw scores only.\footnote{Since
  application track benchmarks may be partially solved, defining a
  useful sequential score for the application track would require
  information not provided by the raw score, e.g., detailed timing
  information for each result.}

Division scores are compared lexicographically:
\begin{itemize}
\item A weighted sum of raw scores $\langle e, n, w, c\rangle$ is
  better than $\langle e', n', w', c'\rangle$ iff $e < e'$ or ($e =
  e'$ and $n > n'$) or ($e = e'$ and $n = n'$ and $w < w'$) or ($e =
  e'$ and $n = n'$ and $w = w'$ and $c < c'$).  That is, fewer errors
  takes precedence over more correct solutions, which takes precedence
  over less wall-clock time taken, which takes precedence over less
  CPU time taken.
\item A weighted sum of sequential scores $\langle e_S, n_S,
  c_S\rangle$ is better than $\langle e_S', n_S', c_S'\rangle$ iff
  $e_S < e_S'$ or ($e_S = e_S'$ and $n_S > n_S'$) or ($e_S = e_S'$ and
  $n_S = n_S'$ and $c_S < c_S'$).  That is, fewer errors takes
  precedence over more correct solutions, which takes precedence over
  less CPU time taken.
\end{itemize}
%
We will not make any comparisons between raw scores and sequential
scores, as these are intended to measure fundamentally different
performance characteristics.

\subsection{Competition-wide scoring (main track)}
\label{sec:competition-wide}

We define a competition-wide metric for the main track, separately for
parallel and sequential performance, as follows.  Let $N_i$ be the
total number of benchmarks in division~$i$ that were used in the
competition (and not removed because of disagreements), and let
$\langle e_i, n_i, w_i, c_i\rangle$ be a solver's raw score for this
division (Section~\ref{sec:division-scoring}).  The solver's
competition-wide raw score is
%
$$\sum_i (e_i == 0 \,?\, (n_i/N_i)^2 : -4) \log_e N_i$$
%
where the sum is over all competitive divisions into which the solver
was entered.\footnote{\emph{Rationale:} This metric purposely
  emphasizes breadth of solver participation---a solver participating
  in many logics need not be the best in any one of them.  The use of
  the square in the metric limits this somewhat---a solver still needs
  to do reasonably well compared to winners to be able to catch up by
  breadth of participation.  The non-linear metric also gives added
  weight to completing close to all benchmarks in a division.

  The constant penalty for errors reflects the fact that any error (in
  a particular division) renders a solver untrustworthy for that
  division.  The value~$4$ balances the community's strong interest in
  correct (thoroughly tested) solvers against the risk of stifling
  innovation: entering a (possibly buggy) solver that can solve all
  benchmarks into a division has a positive expected value if the
  probability of a soundness bug is below $\frac{1}{1+4} = 20$\,\%.
  
  The log scaling is a (somewhat arbitrary) means to adjust the scores
  for the wide variety of numbers of benchmarks in different
  divisions.  It seems a reasonable compromise between linearly
  combining numbers of benchmarks, which would overweigh large
  divisions, and simply summing the fraction of benchmarks solved,
  which would overweigh small divisions.

  The metric is also quite simple, and the metric for a solver is
  independent of the performance of other solvers.  Time is omitted
  from the metric because it is only of third importance in the
  regular competition metric, and is difficult to compare across
  divisions.}  The solver's competition-wide sequential score is
computed from its sequential division scores
(Section~\ref{sec:division-scoring}) according to the same formula.
We will recognize the best three solvers according to these metrics.

\subsection{Other recognitions}

The organizers will also recognize the following contributions:
%
\begin{itemize}
\item \emph{Best new entrant}. The best performing entrant from a new
  solver implementation team, as measured by the competition-wide
  metric.
\item \emph{Benchmarks}. Contributors of new benchmarks.
\end{itemize}
%
The organizers reserve the right to recognize other outstanding
contributions that become apparent in the competition results.

%%%%%%%%%%%%%%%%%%%%%%%%%%%%%%%%%%%%%%%%%%%%%%%%%%%%%%%%%%%%%%%%%%%%%%%%%%%%%%%%

\section{Judging}

The organizers reserve the right, with careful deliberation, to remove
a benchmark from the competition results if it is determined that the
benchmark is faulty (e.g., syntactically invalid in a way that affects
some solvers but not others); and to clarify ambiguities in these
rules that are discovered in the course of the competition.  Authors
of solver entrants may appeal to the organizers to request such
decisions.  Organizers that are affiliated with solver entrants will
be recused from these decisions.  The organizers' decisions are final.

%%%%%%%%%%%%%%%%%%%%%%%%%%%%%%%%%%%%%%%%%%%%%%%%%%%%%%%%%%%%%%%%%%%%%%%%%%%%%%%%

\section{Acknowledgments}

SMT-COMP 2017 is organized under the direction of the SMT Steering
Committee. The organizing team is
%
\begin{itemize}
\setlength{\itemsep}{0pt}
\item \href{http://www.cs.man.ac.uk/~regerg/}{Giles Reger}~--
  University of Manchester, UK (co-organizer)
\item
  \href{https://swt.informatik.uni-freiburg.de/staff/heizmann}{Matthias
    Heizmann}~-- University of Freiburg, Germany (co-organizer)
\item \href{http://user.it.uu.se/~tjawe125/}{Tjark Weber}~-- Uppsala
  University, Sweden (chair)
\end{itemize}
%
Tjark Weber is responsible for policy and procedure decisions, such as
these rules, with input from the co-organizers. He is not associated
with any group creating or submitting solvers.

Many others have contributed benchmarks, effort, and feedback.  Clark
Barrett is maintaining the SMT-LIB benchmark library.  The competition
uses the \href{https://www.starexec.org/}{StarExec} service, which is
hosted at the \href{http://www.cs.uiowa.edu/}{University of Iowa}.
Aaron Stump is providing essential StarExec support.

\header{Disclosure.}
%
Matthias Heizmann is associated with the group producing the
SMTInterpol solver.  Giles Reger is associated with the group
producing the Vampire system.

%%%%%%%%%%%%%%%%%%%%%%%%%%%%%%%%%%%%%%%%%%%%%%%%%%%%%%%%%%%%%%%%%%%%%%%%%%%%%%%%

\bibliographystyle{plain}
\bibliography{biblio}

%%%%%%%%%%%%%%%%%%%%%%%%%%%%%%%%%%%%%%%%%%%%%%%%%%%%%%%%%%%%%%%%%%%%%%%%%%%%%%%%

\end{document}

%% Local Variables:
%% mode: latex
%% mode: flyspell
%% ispell-local-dictionary: "american"
%% LocalWords: arity Heizmann logics Reger satisfiability SMT StarExec Tjark
%% End:
